%------------------------------------------------------------------------------
% Author(s):
%  Brad Bachu
%
% Copyright:
%  Copyright (C) 2020 Brad Bachu, Arjun Mohammed, Nicholas Sammy, Kerry Singh
%
%  This file is part of Applied-Mathematics-Unit2 and is distributed under the
%  terms of the MIT License. See the LICENSE file for details.
%
%  Description: Some conventions and examples how best practices that we require
%     Year: 20XX 
%     Module: X
%     Question: X
%------------------------------------------------------------------------------

You solution should be submitted in a document of this form.

For writing math, the package \texttt{amsmath} package has already been loaded.
This has many features.
For single line equations use the \texttt{equation} environment,
\begin{equation}
	x = 5 \, .
\end{equation}
Note that punctuation was included.
If you have multiple equations use the \texttt{align} environment,
\begin{align}
   x = 5 \, , \\
   y = 4 \,.
\end{align}
For clarity, we do not need to number every equation in an environment, but
we require a number on every block of equations for referencing. To achieve this,
you must use \texttt{\textbackslash nonumber} which has already been shortened to \texttt{\textbackslash nn}.
\begin{align}
   x = 5 \, , \nn \\
   y = 4 \,.
\end{align}

For simplifying, additional punctuation is not needed
\begin{align}
   x &= 3 - 2 \nn \\
   &= 1 \, .
\end{align}

Sometimes, it would be clearer to label subequations.
If $X$ is a random variable with mean $\mu$, then the \textbf{variance}, $Var[X]$, is defined by
\begin{subequations}\label{mod2:eq:Discrete:Variance}
   \begin{align}
      Var[X] &= E[(X-\mu)^2]   \label{mod2:eq:Discrete:Variance:1}  \\
      &= E[X^2] - (E[X])^2  \nn \\
      &= E[X^2] - \mu^2 \, . \label{mod2:eq:Discrete:Variance:2} \  
   \end{align}
   The \texttt{subequations} environment can operate across text also
   \begin{equation}
      Var[X] \equiv \sigma^2 \label{mod2:eq:Discrete:Variance:3}
   \end{equation}
\end{subequations}
The references will work as you'd expect: \ref{mod2:eq:Discrete:Variance:1},
\ref{mod2:eq:Discrete:Variance:2} and \ref{mod2:eq:Discrete:Variance:3} are all part of \ref{mod2:eq:Discrete:Variance}.

All figures and tables should be referenced in the text and should be
placed at the top of the page where they are first cited or in
subsequent pages. Positioning them in the source file
after the paragraph where you first reference them usually yield good
results. See figure~\ref{fig:i} and table~\ref{mod1:tab:LogicSymbols}.
\begin{figure}[tbp]
   \centering % \begin{center}/\end{center} takes some additional vertical space
   \includegraphics[width=.5\textwidth]{images.jpeg}
   \caption{\label{fig:i} Always give a caption.}
\end{figure}

\begin{table}[ht]
   \centering
   \begin{tabular}{|c|c|c|}
      \hline
         Symbol & Name & Read as\\
      \hline
      $\land$ & Conjunction & And \\
      $\lor$ & Disjunction & Or \\
      $\lnot$ & Negation & Not\\
      $\Rightarrow$ & Conditional & If ... then ...\\
      $\Leftrightarrow$ & Bi-conditional & If and only if; iff\\
      \hline
   \end{tabular}
    \caption{\label{mod1:tab:LogicSymbols} We prefer to have borders around the tables.}
\end{table}

All figures must be drawn using a vector graphics. We strongly recommend using
TikZ and have provided common examples in the tikzfigs folder.   

Do not abbreviate and ``table'', 
but ``Eq.'', ``Def'', ``Prop'' and ``Ref.'' are welcome.

We control how subquestions are formatted and thus you must use the
\texttt{subquestions},\texttt{subsubquestions} and \texttt{subsubsubquestions} environments as shown
\begin{subquestions}
   \subquestion This is the first level of a question
      \begin{subsubquestions}
         \subsubquestion Most have another level of questions
         \begin{subsubsubquestions}
            \subsubsubquestion Very few need to go this far
         \end{subsubsubquestions}
         \subsubquestion You can always come out of the environment and continue working
      \end{subsubquestions}
   \subquestion and again move out another level
\end{subquestions}