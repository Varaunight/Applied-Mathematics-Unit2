%------------------------------------------------------------------------------
% Author(s):
% Arjun Mohammed, Brad Bachu
%
% Copyright:
%  Copyright (C) 2020 Brad Bachu, Arjun Mohammed, Nicholas Sammy, Kerry Singh
%
%  This file is part of Applied-Mathematics-Unit2 and is distributed under the
%  terms of the MIT License. See the LICENSE file for details.
%
%  Description:
%     Module 1 notes.
%------------------------------------------------------------------------------


%%%%%%%%%%%%%%%%%%%%%%%%%%%%%%%%%%%%%%%%%%%%%%%%%%%%%%%%%%%%%%%%%%%%%%%%%
% Module 1 Notes and Definitions
%%%%%%%%%%%%%%%%%%%%%%%%%%%%%%%%%%%%%%%%%%%%%%%%%%%%%%%%%%%%%%%%%%%%%%%%%

\section{Linear Programming} \label{Module1:LinearProgramming}

\begin{defn}\label{mod1:defn:TourOfVertices}
	\textbf{Tour of Vertices}: In a 2-dimensional graph, the set of vertices of the feasible region must contain the optimal solution. The Tour of Vertices method checks each vertex of the feasible region to find the optimum solution for the problem.
\end{defn}

\section{Assignment Models} \label{Module1:AsgnmtModels}

\begin{defn} \label{mod1:defn:HungAlgStep4}
	\textbf{Step 4}: This is the step of the Hungarian algorithm that subtracts the matrix minimum from all unshaded elements and adds double the matrix minimum to all elements that are shaded twice. Using this as the reference until the structured Hungarian notes are placed.
\end{defn}

\section{Graph Theory and Critical Analysis} \label{Module1:GraphTheoryCriticalAnalysis}

\textcolor{red} {DIAGRAMS FOR GRAPH DEFINITIONS  AND CRITICAL ANALYSIS ALGORITHMS}\\
\textcolor{red}{HUNGARIAN ALGO}
\begin{defn}\label{mod1:defn:Vertex}
   \textbf{Vertex}: The Vertex is the fundamental unit from which graphs are formed.
\end{defn}

\begin{defn}\label{mod1:defn:Edge}
   \textbf{Edge}: In an undirected graph, an unordered pair of nodes that signify  a line joining  two nodes are said to form an edge. For a directed graph, the edge is an ordered pair of nodes. An edge may also connect a node to itself.
\end{defn}

\begin{defn}\label{mod1:defn:Path}
    \textbf{Path}: A path in a graph is a sequence of edges which joins multiple vertices, in which all vertices (and edges) are distinct.
\end{defn}

\begin{defn}\label{mod1:defn:Degree}
   \textbf{Degree}: The degree of a vertex  in a graph is the number of edges that are incident to the vertex. A loop is counted as two incident edges.
\end{defn}

\begin{defn}\label{mod1:defn:FloatTime}
	\textbf{Float Time}: The float time of an activity in an activity network is the time difference between the latest start time of the activity and the earliest start time of the activity.
\end{defn}

\begin{defn}\label{mod1:defn:CritPath}
	\textbf{Critical Path}: A crtical path is a path which joins activities in an Activity Network whose Float Time is $0$. It is the path which determines the minimum time for some set of operations to be completed. 
\end{defn}
	

\section{Logic and Boolean Algebra}

\begin{defn}\label{mod1:defn:Proposition}
   \textbf{Propsition}: A proposition is a declarative statment which is either true or false.
\end{defn}

\begin{defn}\label{mod1:defn:LogicSymbols}
   Propositional Logical Symbols
      
   \begin{table}[ht]
      \centering
      \begin{tabular}{|c|c|c|}
         \hline
            Symbol & Name & Read as\\
         \hline
         $\land$ & Conjunction & And \\
         $\lor$ &   Disjunction & Or \\
         $\lnot$ & Negation & Not\\
         $\Rightarrow$ & Conditional & If ... then ...\\
         $\Leftrightarrow$ & Bi-conditional & If and only if; iff\\
         \hline
      \end{tabular}
      \label{mod1:tab:LogicSymbols}
   \end{table}

\end{defn}      

\begin{defn}\lbl{mod1:defn:TruthTables}
\begin{table}[!htb]
   \centering
   \begin{minipage}{0.2\textwidth}
   \centering
      \begin{tabular}{|c |c|}
         \hline
         p &  $\lnot$p\\
         \hline
         T & F\\
         F & T\\
         \hline
      \end{tabular}
      \caption{\label{mod1:tab:Negation}Negation}
   \end{minipage}
   \begin{minipage}{0.38\textwidth}
      \centering
      \begin{tabular}{|c c|c|}
            \hline
            p & q & p $\land$ q\\
            \hline
            T & T & T\\
            T & F & F\\
            F & T & F\\
            F & F & F\\
            \hline
      \end{tabular}
      \caption{\label{mod1:tab:Conjunction}Conjunction}
   \end{minipage}
   \vspace{1cm}
   \begin{minipage}{0.38\textwidth}
   \centering
            \begin{tabular}{|c c|c|}
         \hline
         p & q & p $\lor$ q\\
         \hline
         T & T & T\\
         T & F & T\\
         F & T & T\\
         F & F & F\\
         \hline
      \end{tabular}
      \caption{\label{mod1:tab:Disjunction}Disjunction}
   \end{minipage}
   \hspace{6cm}
   \begin{minipage}{0.38\textwidth}
      \centering
      \begin{tabular}{|c c|c|}
         \hline
         p & q & p $\Rightarrow$ q\\
         \hline
         T & T & T\\
         T & F & F\\
         F & T & T\\
         F & F & T\\
         \hline
      \end{tabular}
      \caption{\label{mod1:tab:Conditional}Conditional}
   \end{minipage}
   % \hspace{1cm}
   \begin{minipage}{0.38\textwidth}
      \centering
      \begin{tabular}{|c c|c|}
         \hline
         p & q & p $\Leftrightarrow$ q\\
         \hline
         T & T & T\\
         T & F & F\\
         F & T & F\\
         F & F & T\\
         \hline
      \end{tabular}
      \caption{\label{mod1:tab:Bi-conditional}Bi-conditional}
   \end{minipage}
\end{table}
\end{defn}


\noindent Suppose we have the proposition $p \Rightarrow q$, we make the following three definitions:
\begin{defn}\label{mod1:defn:Inverse}
   \textbf{Inverse}: $\lnot p \Rightarrow \lnot q$ 
\end{defn}

\begin{defn}\label{mod1:defn:Converse}
   \textbf{Converse}: $q \Rightarrow p$ 
\end{defn}

\begin{defn}\label{mod1:defn:Contrapositive}
   \textbf{Contrapositive}: $\lnot q \Rightarrow \lnot p$ 
\end{defn}

\begin{defn}\label{mod1:defn:Tautology}
   \textbf{Tautology}: A tautology is a proposition which is true in every possible interpretation.
\end{defn}

\begin{defn}\label{mod1:defn:Contradiction}
   \textbf{Contradiction}: A contradiction is a proposition which is false in every possible interpretation.
\end{defn}





\subsection{Laws of Boolean Algebra}\label{mod1:section:BooleanAlgebraLaws}



\noindent Consider the propsitional variables $p$, $q$ and $r$


\begin{law}\label{mod1:law:Annulment}
   \textbf{Annulment}
   \begin{align}
   p \land F &= F  \\
   p \lor T &= T 
   \end{align}
\end{law}


\begin{law}\label{mod1:law:Identity}
   \textbf{Identity}
   \begin{align}
   p \land T &= p  \\
   p \lor F &= p
   \end{align}
\end{law}

\begin{law}\label{mod1:law:Idempotent}
   \textbf{Idempotent}
   \begin{align}
   p \lor p &= p\\
   p \land p &= p
   \end{align}
\end{law}

\begin{law}\label{mod1:law:Complement}
   \textbf{Complement}
   \begin{align}
   p \lor \lnot p &= T\\
   p \land \lnot p &= F
   \end{align}
\end{law}

\begin{law}\label{mod1:law:DoubleNegation}
   \textbf{Double Negation}
   \begin{align}
   \lnot(\lnot p) &= p
   \end{align}
\end{law}

\begin{law}\label{mod1:law:DeMorgan}
   \textbf{De Morgan's}
   \begin{align}
   \lnot (p \land q) &= \lnot p \lor \lnot q \\
   \lnot (p \lor q) &= \lnot p \land \lnot q 
   \end{align}
\end{law}

\begin{law}\label{mod1:law:Associative}
   \textbf{Associative}
   \begin{align}
   (p \lor q) \lor r &= p \lor (q \lor r) \\
   (p \land q) \land r &= p \land (q \land r) 
   \end{align}
\end{law}

 \begin{law}\label{mod1:law:Commutative}
   \textbf{Commutative}
   \begin{align}
   p \land q &= q \land p \\
   p \lor q &= q \lor p
   \end{align}
\end{law}

\begin{law}\label{mod1:law:Distributive}
   \textbf{Distributive}
   \begin{align}
   p \land (q \lor r) &= (p \land q) \lor (p \land r) \\
   p \lor (q \land r) &= (p \lor q) \land (p \lor r) 
   \end{align}
\end{law}

\begin{law}\label{mod1:law:Absorptive}
   \textbf{Absorptive}
   \begin{align}
   p \land (p \lor q) &= p\\
   p \lor (p \land q) &= p
   \end{align}
\end{law}


\subsection{Logic Circuits}\label{mod1:section:LogicCircuits}

\begin{defn}
	Logic Circuits can be used to represent Boolean Expressions with switches and wires. Switches are usually defined with logic propositions where a truth value of 1 will indicate a closed switch whereas a truth value of 0 represents an open switch.
\end{defn}

\begin{defn}
	Switching Circuits:
	\begin{figure}[H]
		\centering
		\begin{circuitikz}
			\draw [color=black, thick] (1,0) -- (2,0);
			\draw [color=black, thick] (5,0) -- (4,0);
			\draw (2,0) to[normal open switch, *-*](4,0);	
			\path (2,0) -- (4,0) node[pos=0.5,below]{\textbf{A}};
		\end{circuitikz}
	\caption{Preposition \textbf{A}}
	\end{figure}

	\begin{figure}[H]
		\centering
		\begin{circuitikz}
			\draw [color=black, thick] (1,0) -- (2,0);
			\draw [color=black, thick] (5,0) -- (4,0);
			\draw [color=black, thick] (7,0) -- (8,0);
			
			\draw (2,0) to[normal open switch, *-*](4,0);	
			\path (2,0) -- (4,0) node[pos=0.5,below]{\textbf{X}};
			
			\draw (5,0) to[normal open switch, *-*](7,0);	
			\path (5,0) -- (7,0) node[pos=0.5,below]{\textbf{Y}};
		\end{circuitikz}
		\caption{Preposition (\textbf{X} $\land$ \textbf{Y})}
	\end{figure}

	\begin{figure}[H]
		\centering 
		\begin{circuitikz}
			\draw [color=black, thick] (6,0) -- (7,0);
			\draw [color=black, thick] (7,0) -- (7,1);
			\draw [color=black, thick] (7,0) -- (7,-1);
			\draw [color=black, thick] (9,0) -- (10,0);
			\draw [color=black, thick] (9,0) -- (9,1);
			\draw [color=black, thick] (9,0) -- (9,-1);
			
			
			\draw (7,1) to[normal open switch, *-*](9,1);	
			\path (7,1) -- (9,1) node[pos=0.5,below]{$\sim$\textbf{P}};
			
			\draw (7,-1) to[normal open switch, *-*](9,-1);	
			\path (7,-1) -- (9,-1) node[pos=0.5,below]{\textbf{Q}};
		\end{circuitikz}
		\caption{Preposition ($\sim$\textbf{P} $\lor$ \textbf{Q})}
	\end{figure}
\end{defn}






      
      
      
      
      
      
      
      
      
      
      
      