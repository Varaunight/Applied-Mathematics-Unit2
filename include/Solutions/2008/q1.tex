%------------------------------------------------------------------------------
% Author(s):
% Varaun Ramgoolie
% Copyright:
%  Copyright (C) 2020 Brad Bachu, Arjun Mohammed, Varaun Ramgoolie, Nicholas Sammy
%
%  This file is part of Applied-Mathematics-Unit2 and is distributed under the
%  terms of the MIT License. See the LICENSE file for details.
%
%  Description:
%     Year: 2008 June
%     Module: 1
%     Question: 1 
%------------------------------------------------------------------------------

\begin{subquestions}
	
%------------------------------------------------------------------------------
% 1 a -------------------------------------------------------------------------
%------------------------------------------------------------------------------

\subquestion

The objective function that we want to maximize is,
\begin{equation}
	P=15x + 27y \,. 
\end{equation}

Using $x$ as the number of units of product X and $y$ as the number of units of product Y, our inequalities are,
\begin{align}
	x & \geq 0 \,, \nn \\
	y & \geq 0 \,, \nn \\
	6x + 9y & \leq 360 \,, \nn \\
	15x + 9y & \leq 675 \,.
\end{align}

%------------------------------------------------------------------------------
% 1 b -------------------------------------------------------------------------
%------------------------------------------------------------------------------

\subquestion

The feasible region is shaded in \rfig{2008J:q1:fig:Graph}.

\begin{center}

\begin{figure}
	\begin{center}
		\includegraphics{../2008/figures/2008Jq1Graph}
		\caption{\label{2008J:q1:fig:Graph} Linear Programming Graph.}
	\end{center}
\end{figure}

\end{center}

%------------------------------------------------------------------------------
% 1 c -------------------------------------------------------------------------
%------------------------------------------------------------------------------

\subquestion

From \rdef{mod1:defn:TourOfVertices}, we can calculate $P$ by,
\begin{align}
	\text{Using (0,40)} \,, \nn \\
	P & = 15x + 27y \,, \nn \\
	  & = (15 \times 0) + (27 \times 40) \,, \nn \\
	  & = 1080 \,. \label{2008:q1:eqn:Profit} \\
	\text{Using (35, $\frac{150}{9})$} \,, \nn \\
	P & = 15x + 27y \,, \nn \\
	  & = (15 \times 35) + (27 \times \frac{150}{9}) \,, \nn \\
	  & = 705 \,.    \\		  
	\text{Using (45,0)} \,, \nn \\
	P & = 15x + 27y \,, \nn \\
	  & = (15 \times 45) + (27 \times 0) \,, \nn \\
	  & = 675 \,. 
\end{align}

Thus, from \req{2008:q1:eqn:Profit}, the maximum value of $P$ is $1080$.

\end{subquestions}

