%------------------------------------------------------------------------------
% Author(s):
% Varaun Ramgoolie
% Copyright:
%  Copyright (C) 2020 Brad Bachu, Arjun Mohammed, Varaun Ramgoolie, Nicholas Sammy
%
%  This file is part of Applied-Mathematics-Unit2 and is distributed under the
%  terms of the MIT License. See the LICENSE file for details.
%
%  Description:
%     Year: 2011
%     Module: 2
%     Question: 4 
%------------------------------------------------------------------------------

%------------------------------------------------------------------------------
% 4 a
%------------------------------------------------------------------------------

\begin{subquestions}
	
\subquestion

Let $X$ be the discrete random variable representing the amount of accidents which occur in one week. $X$ follows a Poisson distribution,
\begin{equation}
	X \sim \text{Pois}(3) \,.
\end{equation}
	
\begin{subsubquestions}
	
\subsubquestion

From \rdef{mod2:defn:Poisson}, we know that,
\begin{equation}
	P(X = x) =\frac{ 3 ^ x \times e^{-3}}{x!} \,. \label{2011:q4:Poisson1}
\end{equation}
	
Using this, we can determine $P(X=0)$,
\begin{align}
	P(X=0) & = \frac{ 3 ^0 \times e^{-3}}{0!} \nn \\
	       & = \frac{ 1 \times e^{-3}}{1} \nn \\
	       & = \frac{1}{e^3} \,.
\end{align}
	
%------------------------------------------------------------------------------

\subsubquestion

To determine $P(X>3)$ it is more convenient to write it as,
\begin{align}
	P(X>3) & =1-P(X \leq 3) \,.
\end{align}

Further, since $X$ is discrete,
\begin{align}
	P(X>3)  & = 1 - [P(X=0)+P(X=1)+P(X=2)+P(X=3)] \,. \label{2011:q4:Poisson2}
\end{align}

We use \req{2011:q4:Poisson1} to determine the necessary probabilities, 
\begin{align}
	P(X=0) & = \frac{1}{e^3} \,,
\end{align}
\begin{align}
	P(X=1) & = \frac{ 3^1 \times e^{-3}}{1!} \nn \\
	       & = \frac{3 \times e^{-3}}{1} \nn \\
	       & = \frac{3}{e^3} \,,
\end{align}
\begin{align}
	P(X=2) & = \frac{ 3^2 \times e^{-3}}{2!} \nn \\
	       & = \frac{9 \times e^{-3}}{2} \nn \\
	       & = \frac{9}{2e^3} \,,
\end{align}
and
\begin{align}
	P(X=3) & = \frac{ 3^3 \times e^{-3}}{3!} \nn \\
	       & = \frac{27 \times e^{-3}}{6} \nn \\
	       & = \frac{27}{6e^3} \,. 
\end{align}

Substituting these into \req{2011:q4:Poisson2}, we get,
\begin{align}
	P(X>3) & = 1 - [P(X=0)+P(X=1)+P(X=2)+P(X=3)] \nn \\
			 & = 1 - \left(\frac{1}{e^3} + \frac{3}{e^3} + \frac{9}{2e^3} + \frac{27}{6e^3} \right) \nn \\
		    & = 1 - \frac{13}{e^3} \nn \\
			 & = 0.353 \,.
\end{align}
%------------------------------------------------------------------------------

\subsubquestion

If there are on average 3 accidents in 1 week, there will be on average 6 accidents during a fortnight (2 week period). Let $Y$ be the discrete random variable representing the number of accidents in a 2 week period,
\begin{equation}
	Y \sim \text{Pois}(6) \,.
\end{equation} 

From \rdef{mod2:defn:Poisson}, we know that,
\begin{equation}
	P(Y = y) =\frac{ 6 ^ y \times e^{-6}}{y!} \,.
\end{equation}

Thus, we can find $P(Y=6)$ as,
\begin{align}
	P(Y = 6) & = \frac{ 6 ^ 6 \times e^{-6}}{6!} \nn \\	
	         & = \frac{46656 \times e^{-6}}{720} \nn \\
	         & = \frac{324}{5 \times e^6} \nn \\
	         & = 0.161 \,.
\end{align}

\end{subsubquestions}
	
%------------------------------------------------------------------------------
% 4 b
%------------------------------------------------------------------------------
	
\subquestion

We are given that the probability of success, $p=0.84$. Let $X$ be the discrete random variable representing the number of students that pass an examination. $X$ follows a Binomial distribution,
\begin{equation}
	X \sim \text{Bin}(0.84) \,.
\end{equation} 

From \rdef{mod2:defn:Binomial}, we know that we can determine $P(X=x)$ as,
\begin{equation}
	P(X = x) = { n \choose x} (0.84)^x (1-0.84)^{n-x} \,. \label{2011:q4:Bin1}
\end{equation}

Since two-thirds of 12 is 8, we want to find,
\begin{equation}
	P(X>8) \,.
\end{equation}

Since X is discrete, we know that,
\begin{equation}
	P(X > 8) = P(X=9)+P(X=10)+P(X=11)+P(X=12) \,. \label{2011:q4:Bin2}
\end{equation}

Using \req{2011:q4:Bin1}, we can find each of these probabilities,
\begin{align}
	P(X=9) & = { 12 \choose 9} \times (0.84)^9 \times (1-0.84)^{12-9} \nn \\
	       & = 220 \times (0.84^9) \times (0.16)^3 \nn \\
	       & = 0.188 \,,
\end{align}
\begin{align}
	P(X=10) & = { 12 \choose 10} \times (0.84)^{10} \times (1-0.84)^{12-10} \nn \\
	        & = 66 \times (0.84)^{10} \times (0.16)^2 \nn \\
	        & = 0.296 \,,
\end{align}
\begin{align}
	P(X=11) & = { 12 \choose 11} \times (0.84)^{11} \times (1-0.84)^{12-11} \nn \\
	        & = 12 \times (0.84)^{11} \times 0.16 \nn \\
	        & = 0.282 \,,
\end{align}
and,
\begin{align}
	P(X=12) & = { 12 \choose 12} \times (0.84)^{12} \times (1-0.84)^{12-12} \nn \\
	        & = 1 \times (0.84)^{12} \times 1 \nn \\
	        & = 0.123 \,. 
\end{align}

Hence, substituting these into \req{2011:q4:Bin2}, we get that,
\begin{align}
P(X > 8) & = P(X=9)+P(X=10)+P(X=11)+P(X=12) \nn \\
& = 0.188+0.296+0.282+0.123 \nn \\
& = 0.889 \,.      
\end{align}

%------------------------------------------------------------------------------
% 4 c
%------------------------------------------------------------------------------

\subquestion
 
\begin{subsubquestions}
	
\subsubquestion

The probability that the first marble drawn is white is $\frac{4}{9}$. The probability that the second marble drawn is black is $\frac{3}{8}$. The probability that the third marble drawn is red is $\frac{2}{7}$. Thus, from \rdef{mod2:defn:MultiplicationRule}, 
\begin{align}
	P(white \rightarrow black \rightarrow red) & = \frac{4}{9} \times \frac{3}{8} \times \frac{2}{7} \nn \\
	                                           & = \frac{1}{21} \,.
\end{align}
	
%------------------------------------------------------------------------------

\subsubquestion

Note that the probability obtained in \textbf{(c)(i)} is the same probability as any other arrangement of the \textbf{same set} of colors. 

Thus,
\begin{align}
	P(\text{one of each color}) & = P(white \rightarrow black \rightarrow red) \nn \\
	& \hspace{20mm}\times \nn\\
	&\text{Number or permutations of (W, B, R)} \,.
\end{align}

From \req{mod2:defn:PermutationEqn}, the number of permutations of the \textbf{set of 3} colors is,
\begin{align}
	^3P_3 & = \frac{3!}{(3-3)!} \nn \\
	      & = \frac{3!}{1} \nn \\
	      & = 6 \,.
\end{align}

Hence, we can substitute in the above to find,
\begin{align}
	P(\text{one of each color}) & = P(white \rightarrow black \rightarrow red) \times 6 \nn \\
	                     & = \frac{1}{21} \times 6 \nn \\
	                     & = \frac{2}{7} \,.
\end{align}


%------------------------------------------------------------------------------

\subsubquestion

We will split the marbles into 2 groups: White marbles (W) and not-White marbles (W'). In the setup, there are 4 W marbles and 5 W' marbles. The different combinations of the 3 marbles chosen, if a white is the last marble, are (W,W,W), (W'W,W), (W,W',W) and (W', W',W). 

Using \req{mod2:defn:MultiplicationRule} and \req{mod2:defn:AdditionRule}, we can find that,
\begin{align}
	P(W,W,W) & = \frac{4}{9} \times \frac{3}{8} \times \frac{2}{7} \nn \\
	         & = \frac{1}{21} \,.
\end{align}
\begin{align}
	P(W',W,W) & = \frac{5}{9} \times \frac{4}{8} \times \frac{3}{7} \nn \\
              & = \frac{5}{42} \,.
\end{align}
\begin{align}
	P(W,W',W) & = \frac{4}{9} \times \frac{5}{8} \times \frac{3}{7} \nn \\
			  & = \frac{5}{42} \,. 
\end{align}
\begin{align}
	P(W',W',W)& = \frac{5}{9} \times \frac{4}{8} \times \frac{4}{7} \nn \\
			  & = \frac{10}{63} \,.
\end{align}
\begin{align}
	P(\text{last marble is white}) & = P(W,W,W)+(W',W,W)+P(W,W',W)+P(W',W',W) \nn \\
	& = \frac{1}{21} + \frac{5}{42} + \frac{5}{42} + \frac{10}{63} \nn \\
	                               & = \frac{4}{9} \,.
\end{align}

Alternatively, as we mentioned above, the order of the draws matters, up to an extent. You can convince yourself that the probability that the third marble is white, $P(\text{Anything},\text{Anything},W) =P(W,\text{Anything},\text{Anything})$. Then, we simply need to compute the probability that the first marble is white,
\begin{align}
	P(\text{Anything},\text{Anything},W) &=P(W,\text{Anything},\text{Anything}) \nn\\
	&= P(\text{1st marble is white}) \nn\\
	&= \frac{4}{9}
\end{align}

\end{subsubquestions}

\end{subquestions}