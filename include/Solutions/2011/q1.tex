%------------------------------------------------------------------------------
% Author(s):
% Varaun Ramgoolie
% Copyright:
%  Copyright (C) 2020 Brad Bachu, Arjun Mohammed, Varaun Ramgoolie, Nicholas Sammy
%
%  This file is part of Applied-Mathematics-Unit2 and is distributed under the
%  terms of the MIT License. See the LICENSE file for details.
%
%  Description:
%     Year: 2011
%     Module: 1
%     Question: 1 
%------------------------------------------------------------------------------

\begin{subquestions}

%-----------------------------------------------------------------------------
% 1 a
%-----------------------------------------------------------------------------

\subquestion

From \rtab{2011:q2:tab:TruthTab1}, we see that $(\sim p ~\lor \sim q) \implies (p ~\land \sim q)$ always has the same truth value of $p$.
\begin{table}[ht]
	\centering
	\begin{tabular}{|c|c|c|c|c|c|c|}
		\hline
		p & q & $\sim$ p & $\sim$ q & ($\sim$ p $\lor$ $\sim$ q) & (p $\land$ $\sim$ q) & ($\sim$ p $\lor$ $\sim$ q) $\implies$ (p $\land$ $\sim$ q) \\
		\hline
		0 & 0 & 1 & 1 & 1 & 0 & 0 \\
		0 & 1 & 1 & 0 & 1 & 0 & 0 \\
		1 & 0 & 0 & 1 & 1 & 1 & 1 \\
		1 & 1 & 0 & 0 & 0 & 0 & 1 \\
		\hline
	\end{tabular}
	\caption{\label{2011:q2:tab:TruthTab1} Truth Table of $(\sim p ~\lor \sim q) \implies (p ~\land \sim q)$.}
\end{table}

%-----------------------------------------------------------------------------
% 1 b
%-----------------------------------------------------------------------------

\subquestion

The Boolean expression of the circuit is,
\begin{equation}
	(p \land (q ~\lor \sim r)) \lor r\,.
\end{equation}

%-----------------------------------------------------------------------------
% 1 c
%-----------------------------------------------------------------------------

\subquestion

See the switching circuit for $(a \land b) \lor (a \land (\sim b \lor c))$ below.
\begin{center}
	
\begin{circuitikz}
	\draw [color=black, thin] (0,0) -- (2,0);
	\draw [color=black, thin] (2,0) -- (2,2);
	\draw [color=black, thin] (2,0) -- (2,-2);
	
	\draw (2,2) to[normal open switch, *-*](6,2);
	\draw (6,2) to[normal open switch, *-*](10,2);
	
	\path (2,2) -- (6,2) node[pos=0.5,below]{a};
	\path (6,2) -- (10,2) node[pos=0.5,below]{b};
	
	\draw (2,-2) to[normal open switch, *-*](6,-2);
    \path (2,-2) -- (6,-2) node[pos=0.5,below]{a};

	\draw [color=black, thin] (6,-2) -- (6,-1);
	\draw [color=black, thin] (6,-2) -- (6,-3);
	
	\draw (6,-1) to[normal open switch, *-*](10,-1);
	\draw (6,-3) to[normal open switch, *-*](10,-3);
	
	\path (6,-1) -- (10,-1) node[pos=0.5,below]{$\sim$ b};
	\path (6,-3) -- (10,-3) node[pos=0.5,below]{c};
	
	\draw [color=black, thin] (10,-1) -- (10,-2);
	\draw [color=black, thin] (10,-3) -- (10,-2);
	\draw [color=black, thin] (10,-2) -- (11,-2);
	\draw [color=black, thin] (10,2) -- (11,2);
	\draw [color=black, thin] (11,-2) -- (11,2);
	
	\draw [color=black, thin] (11,0) -- (13,0);
	
\end{circuitikz}

\end{center}
%-----------------------------------------------------------------------------
% 1 d
%-----------------------------------------------------------------------------

\subquestion

\begin{subsubquestions}

\subsubquestion

\begin{subsubsubquestions}
	
\subsubsubquestion

"If there is a west wind, then we shall have rain" can be expressed as,
\begin{equation}
	p \implies q\,.
\end{equation}

%-----------------------------------------------------------------------------

\subsubsubquestion

"If there is no rain, then the west wind does not blow" can be expressed as,
\begin{equation}
	\sim q \implies \sim p\,.
\end{equation}

\end{subsubsubquestions}

%-----------------------------------------------------------------------------

\subsubquestion

From \rtab{2011:q1:tab:TruthTab2}, we can see that $p \implies q$ and $\sim q \implies \sim p$ are logically equivalent.
\begin{table}[ht]
	\centering
	\begin{tabular}{|c|c|c|c|c|c|}
		\hline
		p & q & $\sim$ p & $\sim$ q & p $\implies$ q & $\sim$ q $\implies$ $\sim$ p \\
		\hline
		0 & 0 & 1 & 1 & 1 & 1 \\
	    0 & 1 & 1 & 0 & 1 & 1 \\
	    1 & 0 & 0 & 1 & 0 & 0 \\
	    1 & 1 & 0 & 0 & 1 & 1 \\
	    \hline
	\end{tabular}
	\caption{\label{2011:q1:tab:TruthTab2} Truth values of  $p \implies q$ and $\sim q \implies \sim p$.}
\end{table}

\end{subsubquestions}

%-----------------------------------------------------------------------------
% 1 e
%-----------------------------------------------------------------------------

\subquestion

Using \rdef{mod1:law:DeMorgan}, we can see that,
\begin{equation}	
	\sim((p \land q) ~\lor \sim p)  \equiv ~\sim(p \land q) \land p \,. \label{2012BoolEqn}
\end{equation}

From \rdef{mod1:law:DeMorgan}, we also know that,
\begin{equation}
	\sim(p \land q) \equiv (\sim p ~\lor \sim q) \,.
\end{equation}

Therefore, \req{2012BoolEqn} becomes,
\begin{equation}
	\sim((p \land q) ~\lor \sim p)  \equiv (\sim p ~\lor \sim q) \land p \,.
\end{equation}

Using \rdef{mod1:law:Distributive}, \rdef{mod1:law:Complement} and \rdef{mod1:law:Identity},
\begin{align}
	(\sim p ~\lor \sim q) \land p & \equiv (p ~\land \sim p) \lor (p \land \sim q) \nn \\
	                              & \equiv (F) \lor (p ~\land \sim q) \nn \\
	                              & \equiv (p ~\land \sim q) \,.
\end{align}

Thus,
\begin{equation}
	\sim((p \land q) ~\lor \sim p) \equiv (p ~\land \sim q) \,.
\end{equation}

\end{subquestions}


