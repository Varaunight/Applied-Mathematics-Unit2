%------------------------------------------------------------------------------
% Author(s):
% Varaun Ramgoolie
% 
% Copyright:
%  Copyright (C) 2020 Brad Bachu, Arjun Mohammed, Varaun Ramgoolie, Nicholas Sammy
%
%  This file is part of Applied-Mathematics-Unit2 and is distributed under the
%  terms of the MIT License. See the LICENSE file for details.
%
%  Description:
%     Year: 2015
%     Module: 1
%     Question: 2
%------------------------------------------------------------------------------

\begin{subquestions}
	
\subquestion
	
\begin{subsubquestions}
		
\subsubquestion
		
The Hungarian algorithm is shown in \rtab{2015:q2:tab:HungAlgo}.
		\begin{table}[!hbt]
			\begin{minipage}{0.3\textwidth}
				\centering
				\begin{tabular}{cccc}
					30 & 32 & 31 & 34 \\
					35 & 33 & 30 & 30 \\
					29 & 31 & 28 & 33 \\
					32 & 34 & 29 & 32 \\
				\end{tabular}
				\captionsetup{width=1.1\linewidth}
				\caption*{Matrix From question}
			\end{minipage}
			\hspace{20pt}
			%---------------------------------------------------
			\begin{minipage}{0.3\textwidth}
				\centering
				\begin{tabular}{cccc}
					0 & 2 & 1 & 4 \\
					5 & 3 & 0 & 0 \\
					1 & 3 & 0 & 5 \\
					3 & 5 & 0 & 3 \\
				\end{tabular}
				\captionsetup{width=1.1\linewidth}
				\caption*{Matrix after Reducing Rows}
			\end{minipage}
			\hspace{20pt}
			%---------------------------------------------------
			\begin{minipage}{0.3\textwidth}
				\centering
				\begin{tabular}{cccc}
					0 & 0 & 1 & 4 \\
					5 & 1 & 0 & 0 \\
					1 & 1 & 0 & 5 \\
					3 & 3 & 0 & 3 \\
				\end{tabular}
				\captionsetup{width=1.1\linewidth}
				\caption*{Matrix after Reducing Columns} 
			\end{minipage}
			
			\vspace{20pt} 
			%---------------------------------------------------
			\begin{minipage}{0.3\textwidth}
				\centering
				\begin{tabular} {cccccc}
					&   &        & \hspace{-3.25mm} \hvs{v1} &   &          \\ 
	       \hhs{h1} & 0 &      0 &                         1 & 4 & \hhe[blue]{h1} \\
           \hhs{h2} & 5 &      1 &                         0 & 0 & \hhe[blue]{h2} \\
					& 1 &      1 &                         0 & 5 &          \\
					& 3 &      3 &                         0 & 3 &          \\
					&   &        & \hspace{-3.25mm} \hve[blue]{v1} &   &    \\
				\end{tabular}
				\captionsetup{width=1.1\linewidth}
				\caption*{Shading 0's using the least \\ \centering number of lines}
			\end{minipage}
			\hspace{20pt}
			%---------------------------------------------------
			\begin{minipage}{0.3\textwidth}
				\centering
				\begin{tabular}{cccc}
					  &   &   &   \\
					0 & 0 & 3 & 4 \\
					5 & 1 & 2 & 0 \\
					0 & 0 & 0 & 5 \\
					2 & 2 & 0 & 3 \\
				  	  &   &   &   \\	 
				\end{tabular}
				\captionsetup{width=1.1\linewidth}
				\caption*{Applying Step \ref{mod1:defn:HungAlgStep4} \\ \hspace{0pt}} 
			\end{minipage}
			\hspace{20pt}
			%---------------------------------------------------
			\begin{minipage}{0.3\textwidth}
				\centering
				\begin{tabular} {cccccc}
					&   &        & \hspace{-3.25mm} \hvs{v3}      & \hspace{-3.25mm} \hvs{v4}       &               \\ 
		   \hhs{h3} & 0 &      0 &                         3      &                          4      & \hhe[red]{h3} \\
					& 5 &      1 &                         2      &                          0      &               \\ 
		   \hhs{h4} & 0 &      0 &                         0      &                          5      & \hhe[red]{h4} \\
					& 2 &      2 &                         0      &                          3      &               \\ 
					&   &        & \hspace{-3.25mm} \hve[red]{v3} & \hspace{-3.25mm} \hve[red]{v4}  &               \\
				\end{tabular}
				\captionsetup{width=1.1\linewidth}
				\caption*{Shading 0's using the least \\ \centering number of lines}
			\end{minipage}

			\caption{\label{2015:q2:tab:HungAlgo} Steps of the Hungarian Algorithm.}
		\end{table}	
			
From \rtab{2015:q2:tab:HungAlgo}, we can see that the possible matchings for the teachers are as follows:
		\begin{align}
			&\text{Mr Jones} \rightarrow \text{Class 1, Class 2}\,, \nn \\
			&\text{Mr James} \rightarrow \text{Class 4}\,, \nn \\
			&\text{Mr Wright} \rightarrow \text{Class 1, Class 2, Class 3}\,, \nn \\
			&\text{Mr Small} \rightarrow \text{Class 3}\,. 
		\end{align}
		
Therefore, the matchings to minimize the total time are,
		\begin{align}
			&\text{Mr Jones} \rightarrow \text{Class 1 or Class 2}\,, \nn \\
			&\text{Mr James} \rightarrow \text{Class 4}\,, \nn \\
			&\text{Mr Wright} \rightarrow \text{Class 1 or Class 2}\,, \nn \\
			&\text{Mr Small} \rightarrow \text{Class 3}\,.  
		\end{align}

%----------------------------------------------------------

\subsubquestion

The total time spent in the classroom by the four teachers can be found by summing the individual times of the teachers (based on the result of the Hungarian Algorithm). Evaluating this, we get
		\begin{align}
			\text{Minimum time} = 30+30+31+29 = 120 \, \text{minutes}\,. \\ \nn 
		\end{align}	
	\end{subsubquestions}
	
%--------------------------------------------------------
% 2 b----------------------------------------------------
%--------------------------------------------------------
	
\subquestion
	
The truth table is as follows,
\begin{table}[ht]
		\centering
		\begin{tabular}{|c|c|c|c|}
			\hline
			p & q & p $\land$ q & (p $\land$ q) $\implies$ p \\
			\hline
			0 & 0 & 0 & 1 \\
			0 & 1 & 0 & 1 \\
			1 & 0 & 0 & 1 \\
			1 & 1 & 1 & 1 \\
			\hline
		\end{tabular}
		\caption{\label{2015:q2:tab:TrthTab} Showing the truth values of the expression.}
\end{table}        
	
Looking at \rtab{2015:q2:tab:TrthTab} and using \rdef{mod1:defn:Tautology}, we can see that the expression, $(p \land q) \implies p$, is a \textbf{tautology}.  \\
	
%------------------------------------------------------------------
% 2 c
%------------------------------------------------------------------
		
\subquestion
	
\begin{subsubquestions}
		
\subsubquestion
		
The Boolean expression for the circuit is,
		\begin{align}
			a \land (b \lor (a \land b)) \land (a \lor (\sim a \land b))\,.
		\end{align}	
		
%-----------------------------------------------------------------

\subsubquestion
		
Using \rdef{mod1:law:Associative}, we can manipulate this expression as follows,
		\begin{align}
			a \land (b \lor (a \land b)) \land (a \lor (\sim a \land b))
			 \equiv a \land (a \lor (\sim a \land b)) \land (b \lor (a \land b))\,, \label{2015:q2:BooleanEquation} \nn \\
		\end{align}
		
Let $A \equiv a \land (a \lor (\sim a \land b))$ and let $B \equiv (b \lor (a \land b))$. \\

Using \rdef{mod1:law:Absorptive} on $A$, we see that,
\begin{align}
	A & \equiv a \land (a \lor (\sim a \land b)) \nn \\
	  & \equiv a \,.
\end{align}

Similarly, using \rdef{mod1:law:Associative} and \rdef{mod1:law:Absorptive} on $B$, we see that,
\begin{align}
	B & \equiv (b \lor (a \land b)) \nn \\
	  & \equiv (b \lor (b \land a)) \nn \\
	  & \equiv b \,.
\end{align}

Therefore, \req{2015:q2:BooleanEquation} becomes,
\begin{align}
	a \land (b \lor (a \land b)) \land (a \lor (\sim a \land b)) \equiv a \land b \,.
\end{align}

The expression, $a \land b$, can be expressed as a switching circuit as follows, \\
\begin{center}	
	\begin{circuitikz}
			\draw [thick] (1,0) to (2,0);
			\draw (2,0) to [nos](3,0);
			\path (2,0) -- (3,0) node[pos=0.5, below]{$a$};
			\draw [thick] (3,0) to (5,0);
			\draw (5,0) to [nos](6,0);
			\path (5,0) -- (6,0) node[pos=0.5, below]{$b$};
			\draw [thick] (6,0) to (7,0);
	\end{circuitikz}
\end{center}
		
\end{subsubquestions}
	
\end{subquestions}

