%------------------------------------------------------------------------------
% Author(s): Varaun Ramgoolie
%
% Copyright:
%  Copyright (C) 2020 Brad Bachu, Arjun Mohammed, Nicholas Sammy, Kerry Singh
%
%  This file is part of Applied-Mathematics-Unit2 and is distributed under the
%  terms of the MIT License. See the LICENSE file for details.
%
%  Description:
%     Year: 2015
%     Module: 2
%     Question: 3
%------------------------------------------------------------------------------

\begin{subquestions}
	
%------------------------------------------------------------------------
% 3 (a)
%------------------------------------------------------------------------
\subquestion

Let the probability a bolt is defective, $P(D)= 0.2$. 

\begin{subsubquestions}
	
   %-------------------------------------------------------------------------------------
   % 3 (a) i
   %------------------------------------------------------------------------------------
   \subsubquestion	

   Let $X$ be the discrete random variable representing ``the number of defective bolts in a random sample of size 15''. From Def.~\ref{mod2:defn:Binomial}, we see that $X$ can be modelled using a Binomial Distribution,
   \begin{equation}
   	 X \sim \text{Bin}(15,0.2) \,,
   \end{equation}
   and
   \begin{equation}
   	P(X = x) = { 15 \choose x} 0.2^x (1-0.2)^{15-x} \,. 
   \end{equation}

   We want to find $P(X>1)$. To solve this question, we can either find the sum of the different probabilities for defective bolts from 2 to 15 OR we can find the probability that the number of defective bolts is $\leq$ 1. This is because $P(X>1)=1-P(X\leq1)$.  \\

   Using the latter, we see that it is much simpler to calculate $P(X\leq1) $   by noticing that 
   \begin{equation}
   	P(X \leq 1)  =  P(X=1)+P(X=0). 
   \end{equation}

   Solving for each probability, we get
   \begin{align}
   	P(X=0)&={ 15 \choose 0} 0.2^0 (1-0.2)^{15-0} \nn \\
   	&=1\times1\times0.8^{15} \nn \\
   	&=0.8^{15}\,,
   \end{align}
   and,
   \begin{align}
   	P(X=1)&={ 15 \choose 1} 0.2^1 (1-0.2)^{15-1} \nn \\
   	&=15 \times 0.2 \times 0.8^{14} \nn \\
   	&=3\times 0.8^{14} \,. 
   \end{align}

   Substituting, in the above we find,
   \begin{align}
   	P(X>1)&=1-P(X\leq1) \nn \\
   	&=1-(P(X=0)+P(X=1)) \nn \\
   	&=1-(0.8^{15}+3\times0.8^{14}) \nn \\
   	&=1- 0.167 \nn \\
   	&=0.833 \, .
   \end{align}

   %------------------------------------------------------------------------
   % 3 a ii
   %--------------------------------------------------------------------------
   \subsubquestion

   We now want to find the smallest value of $n$ such that,
   \begin{equation}
   	\frac{\sqrt{\var[X]}}{E[X]}< 0.1 \,. 
   \end{equation}

   We can substitute the definitions for $E(X)$ and $\var(X)$ using Eqs.~\ref{mod2:eq:Binomial:Mean} and Eq.~\ref{mod2:eq:Binomial:Variance},
   \begin{align}
   	\frac{\sqrt{np(1-p)}}{np}&<0.1 \\
       \implies \sqrt{\frac{np(1-p)}{n^2p^2}}&<0.1.  \\ 
   \end{align}

   Simplifying this expression yields,
   \begin{equation}
   	\sqrt{\frac{1-p}{np}}<0.1 \, .
   \end{equation}

   Now, we evaluate the expression and solve for $n$,
   \begin{align}
   		&\implies \sqrt{\frac{0.8}{0.2n}}<0.1 \nn \\
   		&\implies \sqrt{\frac{4}{n}}<0.1 \nn \\
   		&\implies \frac{2}{\sqrt{n}}<0.1 \nn \\
   		&\implies \frac{2}{0.1}<\sqrt{n} \nn \\
   		&\implies 20<\sqrt{n}\, . 
   \end{align}

   Lastly, we square the expression to give
   \begin{equation}
   	400<n.  
   \end{equation}

   We get that $n$ must be an integer that is greater than $400$. Therefore, the smallest value of $n$ that satisfies our given criteria is $n=401$.
\end{subsubquestions}  

%------------------------------------------------------------------------  
% 3 b
%------------------------------------------------------------------------
\subquestion

In order to find $P(X\leq2)$, we can use a Poisson approximation to the binomial distribution. \\ \\
From Def.~\ref{mod2:defn:PoissonApproxBinomial}, we know that certain conditions must hold for a binomial distribution to be approximated by a Poisson distribution.
From the given distribution, we can see that $n=500>50$ and $np=2.5<5$, and so the following approximation holds,
\begin{equation}
   X \sim \text{Pois} (2.5) \,.
\end{equation}

Now, we want to find $P(X\leq2)$. For a Poisson Distribution, 
\begin{align}
	P(X \leq 2) = P(0) + P(1) + P(2) \,.
\end{align}

By using Eq.~\ref{mod2:eq:PoissonDist}, we can evaluate each of the probabilities, 
\begin{align}
	P(X = 0) & =\frac{ 2.5 ^ 0 \times e^{-2.5}}{0!} \nn \\
  & =\frac{1 \times e^{-2.5}}{1} \nn \\
  & =e^{-2.5} \nn \\
  & = 0.082 \,,
\end{align}

\begin{align}
	P(X = 1) & =\frac{ 2.5 ^ 1 \times e^{-2.5}}{1!} \nn \\
	& =\frac{2.5 \times e^{-2.5}}{1} \nn \\
	& =2.5 \times e^{-2.5} \nn \\
	& = 0.205 \,,
\end{align}
and,
\begin{align}
	P(X = 2) & =\frac{ 2.5 ^ 2 \times e^{-2.5}}{2!} \nn \\
	& =\frac{2.5^2 \times e^{-2.5}}{2} \nn \\
	& = 0.257\,.
\end{align}
	
Finally, substituting for $P(X \leq 2)$, we get 
\begin{align}
		P(X \leq 2) & = 0.082 + 0.205 + 0.257 \nn \\
		& = 0.544 \,.
\end{align}

%------------------------------------------------------------------------  
% 3 c
%------------------------------------------------------------------------
\subquestion

In order to find, $P(X\leq 25)$, we must use a normal approximation to the Poisson distribution as seen in Def \ref{mod2:def:NormalApproxToPois:Definition}.\\

Therefore, by Note \ref{mod2:note:NormalApproxtoPois:Continuity} and using a continuity correction, we will find the probabilty that $P(X \leq 25.5)$.

Standardizing using Property \ref{mod2:note:Normal:Property2} in Note \ref{mod2:note:Normal:Properties}, we evaluate the probability as follows,
\begin{align}
	P(X \leq 25.5) &= P(Z< \frac{25.5-20}{\sqrt{20}}) \nn \\
	&=P(Z<1.23) \nn \\ 
	&=0.891 \,.
\end{align}

%------------------------------------------------------------------------  
% 3 d
%------------------------------------------------------------------------
\subquestion

To find $P(X<4)$, we can either sum all the individual probabilities from $X=1$ to $3$ OR we can find $P(X>3)$ because $P(X<4)=1-P(X>3)$.

Using the property of \ref{mod2:eq:Geometric:Prop},
\begin{equation}
   P(X>3) = 0.7^3\,,
\end{equation}
we can determine $P(X<4)$ as follows,
\begin{align}
	P(X<4) &= 1-P(X>3) \nn \\
	&=1-0.7^3 \nn \\
	&=1-0.343 \nn \\
	&=0.657 \,.
\end{align}

\end{subquestions}
