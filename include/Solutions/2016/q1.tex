%------------------------------------------------------------------------------
% Author(s):
% Varaun Ramgoolie
% Copyright:
%  Copyright (C) 2020 Brad Bachu, Arjun Mohammed, Varaun Ramgoolie, Nicholas Sammy
%
%  This file is part of Applied-Mathematics-Unit2 and is distributed under the
%  terms of the MIT License. See the LICENSE file for details.
%
%  Description:
%     Year: 2016
%     Module: 1
%     Question: 1
%------------------------------------------------------------------------------

\begin{subquestions}

%------------------------------------------------------------------------------
% 1 a -------------------------------------------------------------------------
%------------------------------------------------------------------------------

\subquestion

\begin{subsubquestions}
	
\subsubquestion
We wish to show that the the proposition $\boldsymbol{(p ~\land \sim q) \lor (\sim p \land q)}$ is equivalent to proposition $\boldsymbol{(p \lor q) \land (\sim p ~\lor \sim q)}$. We can construct the following truth tables to determine whether these propositions are equivalent.
\begin{table}[H]
	\centering
	\begin{tabular}{|c|c|c|c|c|c|c|}
		\hline
		$\boldsymbol{p}$ & $\boldsymbol{q}$ & $\boldsymbol{\sim p}$ & $\boldsymbol{\sim q}$ & $\boldsymbol{(p \land \sim q)}$ & $\boldsymbol{(\sim p \land q)}$ & $\boldsymbol{(p \ \land \sim q) \lor (\sim p \ \land q)}$ \\
		\hline
		0 & 0 & 1 & 1 & 0 & 0 & 0 \\
		0 & 1 & 1 & 0 & 0 & 1 & 1  \\ 
		1 & 0 & 0 & 1 & 1 & 0 & 1  \\
		1 & 1 & 0 & 0 & 0 & 0 & 0  \\
		\hline
	\end{tabular}
	\caption{\label{2016:q1:tab:TruthTab1} Truth Table of $\boldsymbol{(p ~\land \sim q) \lor (\sim p \ \land q)}$.}
\end{table}

\begin{table}[H]
	\centering
	\begin{tabular}{|c|c|c|c|c|c|c|}
		\hline
		$\boldsymbol{p}$ & $\boldsymbol{q}$ & $\boldsymbol{\sim p}$ & $\boldsymbol{\sim q}$ & $\boldsymbol{(p \lor q)}$ & $\boldsymbol{(\sim p \ \lor \sim q)}$ & $\boldsymbol{(p \lor q) \land (\sim p \ \lor \sim q)}$ \\
		\hline
		0 & 0 & 1 & 1 & 0 & 1 & 0  \\
		0 & 1 & 1 & 0 & 1 & 1 & 1  \\ 
		1 & 0 & 0 & 1 & 1 & 1 & 1  \\
		1 & 1 & 0 & 0 & 1 & 0 & 0  \\
		\hline
	\end{tabular}
	\caption{\label{2016:q1:tab:TruthTab1} Truth Table of $\boldsymbol{(p \lor q) \land (\sim p \ \lor \sim q)}$.}
\end{table}

%------------------------------------------------------------------------------

\subsubquestion

Using the Distributive (\rdef{mod1:law:Distributive}) law three times, the Complement law (\rdef{mod1:law:Complement}) and the Identity law (\rdef{mod1:law:Identity}), we can see that,
\begin{align}
	\boldsymbol{(p \ \land \sim q) \lor (\sim p \land q)}
	& \equiv \boldsymbol{(( p \ \land \sim q) ~\lor \sim p) \land ((p \ \land \sim q) \lor q)} \nn \\
	& \equiv \boldsymbol{((p \ \lor \sim p) \land (\sim q \ \lor \sim p)) \land ((p \lor q) \land (\sim q \lor q))} \nn \\
	& \equiv \boldsymbol{T \land (\sim q \ \lor \sim p) \land (q \lor p) \land T} \nn \\
	& \equiv \boldsymbol{(q \ \lor \sim p) \land (q \lor p)} \, .
\end{align}

Therefore,
\begin{align}
	\boldsymbol{(p ~\land \sim q) \lor (\sim p \land q) \equiv (\sim p ~\lor \sim q) \land (q \lor p)} \,.
\end{align}

\end{subsubquestions}

%------------------------------------------------------------------------------
% 1 b -------------------------------------------------------------------------
%------------------------------------------------------------------------------

\subquestion

\begin{subsubquestions}
	
\subsubquestion

We can draw the switching circuit of $\boldsymbol{(p ~\land \sim q) \lor (\sim p ~\land q)}$ below.
\begin{center}
\begin{circuitikz}[scale=0.75]
	\draw [color=black, thin] (0,0) -- (2,0);
	\draw [color=black, thin] (2,0) -- (2,2);
	\draw [color=black, thin] (2,0) -- (2,-2);
	
	\draw (2,2) to[normal open switch, *-*](6,2);
	\draw (6,2) to[normal open switch, *-*](10,2);
	
	\path (2,2) -- (6,2) node[pos=0.5,below]{$p$};
	\path (6,2) -- (10,2) node[pos=0.5,below]{$\sim q$};
	
	\draw (2,-2) to[normal open switch, *-*](6,-2);
	\draw (6,-2) to[normal open switch, *-*](10,-2);
	
	\path (2,-2) -- (6,-2) node[pos=0.5,below]{$\sim p$};
	\path (6,-2) -- (10,-2) node[pos=0.5,below]{$q$};

	\draw [color=black, thin] (10,2) -- (10,0);
	\draw [color=black, thin] (10,-2) -- (10,0);
	\draw [color=black, thin] (10,0) -- (12,0);
	
\end{circuitikz}

\end{center}

%------------------------------------------------------------------------------

\subsubquestion

We can draw the switching circuit of  $\boldsymbol{(p \lor q) \land (\sim p \lor \sim q)}$ below.

\begin{center}
\begin{circuitikz} [scale=0.75]
	\draw [color=black, thin] (0,0) -- (2,0);
	\draw [color=black, thin] (2,0) -- (2,2);
	\draw [color=black, thin] (2,0) -- (2,-2);
	
	\draw (2,2) to[normal open switch, *-*](6,2);
	\draw (2,-2) to[normal open switch, *-*](6,-2);
	
	\path (2,2) -- (6,2) node[pos=0.5,below]{$p$};
	\path (2,-2) -- (6,-2) node[pos=0.5,below]{$q$};
	
	\draw [color=black, thin] (6,2) -- (6,0);
	\draw [color=black, thin] (6,-2) -- (6,0);
	
	\draw [color=black, thin] (6,0) -- (8,0);
	\draw [color=black, thin] (8,0) -- (8,2);
	\draw [color=black, thin] (8,0) -- (8,-2);
	
	\draw (8,2) to[normal open switch, *-*](12,2);
	\draw (8,-2) to[normal open switch, *-*](12,-2);
	
	\path (8,2) -- (12,2) node[pos=0.5,below]{$\sim p$};
	\path (8,-2) -- (12,-2) node[pos=0.5,below]{$\sim q$};
	
	\draw [color=black, thin] (12,2) -- (12,0);
	\draw [color=black, thin] (12,-2) -- (12,0);
	\draw [color=black, thin] (12,0) -- (14,0);
	
\end{circuitikz}

\end{center}

\end{subsubquestions}

%------------------------------------------------------------------------------
% 1 c -------------------------------------------------------------------------
%------------------------------------------------------------------------------

\subquestion

We can write the following Boolean expression to represent the output of the given switching circuit.
\begin{equation}
	\boldsymbol{(p ~\land \sim q) \lor (p ~\land \sim r)}\,.
\end{equation}

%------------------------------------------------------------------------------
% 1 d -------------------------------------------------------------------------
%------------------------------------------------------------------------------

\subquestion
Given the following propositions, \\
$\boldsymbol{p}$: A person who eats red meat \\
$\boldsymbol{q}$: A person who has a high cholesterol reading \\
$\boldsymbol{r}$: A person with a normal cholesterol reading \\
$\boldsymbol{s}$: A person who suffers a heart attack \\

\begin{subsubquestions}
	
\subsubquestion

The Boolean expression for "If a person eats red meat then that person may have a high cholesterol reading or suffer a heart attack." can be expressed as,
\begin{equation}
	\boldsymbol{p \implies (q \lor s)}\,.
\end{equation}

%------------------------------------------------------------------------------

\subsubquestion

The Boolean expression for "If a person does not suffer a heart attack then that person has a normal cholesterol reading and does not eat red meat." can be expressed as,
\begin{equation}
	\boldsymbol{\sim s \implies (r ~\land \sim p)}\,.
\end{equation}

%------------------------------------------------------------------------------

\subsubquestion

We can create the truth table for the proposition $\boldsymbol{p \implies (q \lor s)}$.
\begin{table}[ht]
	\centering
	\begin{tabular}{|c|c|c|c|c|}
		\hline
		$\boldsymbol{p}$ & $\boldsymbol{q}$ & $\boldsymbol{s}$ & $\boldsymbol{q \lor s}$ & $\boldsymbol{p \implies (q \lor s)}$ \\
		\hline
		0 & 0 & 0 & 0 & 1 \\
		0 & 0 & 1 & 1 & 1 \\
		0 & 1 & 0 & 1 & 1 \\
		0 & 1 & 1 & 1 & 1 \\
		1 & 0 & 0 & 0 & 0 \\
		1 & 0 & 1 & 1 & 1 \\
		1 & 1 & 0 & 1 & 1 \\
		1 & 1 & 1 & 1 & 1 \\
		\hline
	\end{tabular}
	\caption{\label{2016:q1:tab:TruthTab2} Truth Table of $\boldsymbol{p \implies (q \lor s)}$\,.}
\end{table}

From \rtab{2016:q1:tab:TruthTab2} and \rdef{mod1:defn:Tautology}, the propsition $\boldsymbol{p \implies (q \lor s)}$ is not a tautology as its value is not always 1.

\end{subsubquestions}

\end{subquestions}

