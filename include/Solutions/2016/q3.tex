%------------------------------------------------------------------------------
% Author(s):
% Varaun Ramgoolie
% Copyright:
%  Copyright (C) 2020 Brad Bachu, Arjun Mohammed, Varaun Ramgoolie, Nicholas Sammy
%
%  This file is part of Applied-Mathematics-Unit2 and is distributed under the
%  terms of the MIT License. See the LICENSE file for details.
%
%  Description:
%     Year: 2016
%     Module: 2
%     Question: 3
%------------------------------------------------------------------------------

%------------------------------------------------------------------------------
% 3 a
%------------------------------------------------------------------------------

\begin{subquestions}
	
\subquestion

We are given that there are 10 water-color paintings and 7 oil paintings available to students to create a portfolio of 12 paintings.	
	
\begin{subsubquestions}
	
\subsubquestion

In total, there are 17 paintings available. Thus, using \rdef{mod2:defn:CombinationEqn}, the number of portfolios that can be created is,
\begin{align}
	^{17}C_{12} & = \frac{17!}{(17-12)! \times 12!} \nn \\
	            & = \frac{17!}{5! \times 12!} \nn \\
	            & = 6188 \,.
\end{align}
	
%------------------------------------------------------------------------------
	
\subsubquestion 

In order to find the number of portfolios with 8 water-color and 4 oil paintings, we must first find the number of ways to choose 8 water-color paintings from 10 paintings and the number of ways to choose 4 oil paintings from 7 paintings.

For the 8 water-color paintings,
\begin{align}
	^{10}C_{8} & = \frac{10!}{(10-8)! \times 8!} \nn \\
	           & = \frac{10!}{2! \times 8!} \nn \\
	           & = 45 \,.
\end{align}

For the 4 oil paintings,
\begin{align}
	^{7}C_{4} & = \frac{7!}{(7-4)! \times 4!} \nn \\
	          & = \frac{7!}{3! \times 4!} \nn \\
	          & = 35 \,.
\end{align}

Therefore, using \rdef{mod2:defn:MultiplicationRule}, the number of portfolios with 8 water-color \textbf{and} 4 oil paintings is,
\begin{align}
	^{10}C_{8} ~\times ~ ^7C_{4} & = 45 \times 35 \nn \\
	                            & = 1575 \,.
\end{align}

%------------------------------------------------------------------------------

\subsubquestion

Given that the selection of paintings are random, the probability that 8 water-color and 4 oil paintings are chosen can be calculated by dividing the number of portfolios with 8 water-color and 4 oil paintings by the total number of portfolios. This would be,
\begin{align}
	P(\text{8 water-color and 4 oil paintings}) & = \frac{\text{Portfolios with 8 water-color and 4 oil paintings}}{\text{Total number of portfolios}} \nn \\
	                                            & = \frac{1575}{6188} \nn \\
	                                            & = \frac{225}{884} \nn \\
	                                            & = 0.255 \,.
\end{align}

\end{subsubquestions}
	
%------------------------------------------------------------------------------
% 3 b
%------------------------------------------------------------------------------

\subquestion

It is given that a player must roll a 6 on a fair die in order to start a certain game. We know that the probability of rolling a 6 on a fair die is $\frac{1}{6}$. Let a discrete random variable $X$ be "the number of rolls a player does before starting a game".
\end{subquestions}