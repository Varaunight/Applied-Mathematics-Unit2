%------------------------------------------------------------------------------
% Author(s):
% Varaun Ramgoolie
% Copyright:
%  Copyright (C) 2020 Brad Bachu, Arjun Mohammed, Varaun Ramgoolie, Nicholas Sammy
%
%  This file is part of Applied-Mathematics-Unit2 and is distributed under the
%  terms of the MIT License. See the LICENSE file for details.
%
%  Description:
%     Year: 2010
%     Module: 2
%     Question: 3 
%------------------------------------------------------------------------------

%------------------------------------------------------------------------------
% 3 a
%------------------------------------------------------------------------------

\begin{subquestions}
	
\subquestion

The cumulative density function, $F$, of some continuous random variable $X$ is given.

\begin{subsubquestions}
	
\subsubquestion

From \rdef{mod2:defn:ContinuousRandomVar}, we know that,
\begin{equation}
	F(x) = P(X \leq x) \,.
\end{equation}
	
From the given $F(x)$, we know that,
\begin{equation}
	F(6) = 1 \,.
\end{equation}

Thus, we get that,
\begin{align}
	F(6) = k(x-3) & = 1 \nn \\
	       k(6-3) & = 1 \nn \\
	       3k & = 1 \nn \\
	       \implies k & = \frac{1}{3} \,.
\end{align}

\rfig{2010:q3:fig:FGraph} shows $y=F(x)$.
\begin{figure}[H]
	\begin{center}
		\includegraphics{../2010/figures/2010q3-a-i}
		\caption{\label{2010:q3:fig:FGraph} Graph of $y=F(x)$.}
	\end{center}
\end{figure}

%------------------------------------------------------------------------------

\subsubquestion

Using Note ~\ref{mod2:note:ContinuousRV:CDF}, we can see that,
\begin{align}
	P(3.5 \leq X \leq 5) & = F(5) - F(3.5) \nn \\
	                     & = \frac{1}{3}(5-3) - \frac{1}{3}(3.5-3) \nn \\
	                     & = \frac{2}{3} - \frac{1}{6} \nn \\
	                     & = \frac{1}{2} \,.
\end{align}

%------------------------------------------------------------------------------

\subsubquestion

From \rprop{mod2:prop:ContinuousRV:CDF}, we know that,
\begin{equation}
	F(M) = \frac{1}{2} \,.
\end{equation}

Thus, the median of $X$ can be found as,
\begin{align}
	F(M) = \frac{1}{3}(M-3) & = \frac{1}{2} \nn \\
	       \implies (M-3) & = 3 \times \frac{1}{2} \nn \\
	       \implies M & = \frac{3}{2} + 3 \nn \\
	                  & = \frac{9}{2} \,.
\end{align}

%------------------------------------------------------------------------------

\subsubquestion

From \rprop{mod2:prop:ContinuousRV:CDF}, we know that,
\begin{equation}
	F(LQ) = \frac{1}{4} \,.
\end{equation}

Thus, the lower quartile of $X$ can be found as,
\begin{align}
	F(LQ) = \frac{1}{3}(LQ-3) & = \frac{1}{4} \nn \\
			 \implies (LQ-3) & = 3 \times \frac{1}{4} \nn \\
			     \implies LQ & = \frac{3}{4} + 3 \nn \\
				      	     & = \frac{15}{4} \,.
\end{align}

\end{subsubquestions}

%------------------------------------------------------------------------------
% 3 b
%------------------------------------------------------------------------------
	
\subquestion

\begin{subsubquestions}
	
\subsubquestion

From \rdef{mod2:defn:ContinuousRV:CDF}, we see that,
\begin{equation}
	f(x) = \frac{d}{dx}F(x) \,.
\end{equation}
	
Thus, we find $f(x)$ as,
\begin{align}
	f(x) & = \frac{d}{dx}F(x) \nn \\
	     & = \frac{d}{dx}\left(\frac{1}{3}(x-3) \right) \nn \\
	     & = \frac{1}{3} \,.
\end{align}

\rfig{2010:q3:fig:fGraph} shows $y=f(x)$.
\begin{figure}[H]
	\begin{center}
		\includegraphics{../2010/figures/2010q3-b-i}
		\caption{\label{2010:q3:fig:fGraph} Graph of $y=f(x)$.}
	\end{center}
\end{figure}

We should note that,
\[ f(x) =\begin{cases} 
	0, & x \leq 3 \\
	\frac{1}{3}, & 3 \leq x \leq 6 \\
	0, & x \geq 6 
		\end{cases}
\]

%------------------------------------------------------------------------------

\subsubquestion

Using \rdef{mod2:defn:ContinuousRV:Expectation}, we can find that E(X) is,
\begin{align}
	E(X) & = \int_{-\infty}^{\infty}x f(x)\mathrm{d}x \nn \\
		 & = \int_{3}^{6}x f(x)\mathrm{d}x \nn \\
	     & = \int_{3}^{6} \left(x \times \frac{1}{3} \right)\mathrm{d}x\nn \\
	     & = \left[\frac{x^2}{6} \right]^{6}_{3} \nn \\
	     & = \left( \frac{6^2}{6} \right) - \left( \frac{3^2}{6} \right) \nn \\
	     & = 6 - \frac{9}{6} \nn \\
	     & = 4.5 \,.
\end{align}

%------------------------------------------------------------------------------

\subsubquestion

Using \rdef{mod2:defn:ContinuousRV:Variance}, we know that Var(X) is,
\begin{align}
	\var(X) & = E(X^2) - {E(X)}^2, \nn \\ \label{2010:q3:Var}
\end{align}

We then proceed by finding,
\begin{align}
	E(X^2)& = \int_{-\infty}^{\infty}x^2 f(x).\mathrm{d}x \nn \\
	& = \int_{3}^{6} \left( x^2 \times \frac{1}{3} \right).\mathrm{d}x \nn \\
	& = \left[\frac{x^3}{9} \right]^{6}_{3} \nn \\
	& = \left( \frac{6^3}{9} \right) - \left( \frac{3^3}{9} \right) \nn \\
	& = 24 -3=21 \,.
\end{align}

Thus, substituting our values into \req{2010:q3:Var},
\begin{align}
\var(X) & = E(X^2) - {E(X)}^2 \nn \\
& = 21 - 4.5^2 \nn \\
& = \frac{3}{4} \,.
\end{align}

\end{subsubquestions}

\end{subquestions}