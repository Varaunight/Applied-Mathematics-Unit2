%------------------------------------------------------------------------------
% Author(s):
% Varaun Ramgoolie
% Copyright:
%  Copyright (C) 2020 Brad Bachu, Arjun Mohammed, Varaun Ramgoolie, Nicholas Sammy
%
%  This file is part of Applied-Mathematics-Unit2 and is distributed under the
%  terms of the MIT License. See the LICENSE file for details.
%
%  Description:
%     Year: 2010
%     Module: 2
%     Question: 3 
%------------------------------------------------------------------------------

%------------------------------------------------------------------------------
% 3 a
%------------------------------------------------------------------------------

\begin{subquestions}
	
\subquestion

The cumulative density function, $F$, of some continuous random variable $X$ is given.

\begin{subsubquestions}
	
\subsubquestion

From \rdef{mod2:defn:ContinuousRandomVar}, recall that,
\begin{equation}
	F(x) = P(X \leq x) \,.
\end{equation}
	
From the given $F(x)$, we know that,
\begin{equation}
	F(6) = 1 \,.
\end{equation}

Substituting in the given expression, we find,
\begin{align}
	F(6) = k(x-3) & = 1 \nn \\
	       k(6-3) & = 1 \nn \\
	       3k & = 1 \nn \\
	       \implies k & = \frac{1}{3} \,.
\end{align}

\rfig{2010:q3:fig:FGraph} shows $y=F(x)$.
\begin{figure}[H]
	\begin{center}
		\includegraphics{../2010/figures/2010q3-a-i}
		\caption{\label{2010:q3:fig:FGraph} Graph of $y=F(x)$.}
	\end{center}
\end{figure}

%------------------------------------------------------------------------------

\subsubquestion

We can compute probabilites from the cumulative density function using Note ~\ref{mod2:note:ContinuousRV:CDF}, 
\begin{align}
	P(3.5 \leq X \leq 5) & = F(5) - F(3.5)  \,.
\end{align}
Substituting, we find,
\begin{align}
	P(3.5 \leq X \leq 5) & = \frac{1}{3}(5-3) - \frac{1}{3}(3.5-3) \nn \\
	                     & = \frac{2}{3} - \frac{1}{6} \nn \\
	                     & = \frac{1}{2} \,.
\end{align}

%------------------------------------------------------------------------------

\subsubquestion

Recall \rprop{mod2:prop:ContinuousRV:CDF},that the median $M$, is the value $x=M$ such that,
\begin{equation}
	F(M) = \frac{1}{2} \,.
\end{equation}

Thus, the median of $X$ can be found as,
\begin{align}
	F(M) = \frac{1}{3}(M-3) & = \frac{1}{2} \nn \\
	       \implies (M-3) & = 3 \times \frac{1}{2} \nn \\
	       \implies M & = \frac{3}{2} + 3 \nn \\
	                  & = \frac{9}{2} \,.
\end{align}

%------------------------------------------------------------------------------

\subsubquestion

From \rprop{mod2:prop:ContinuousRV:CDF}, we know that, the $LQ$ is the value of $x=LQ$ such that,
\begin{equation}
	F(LQ) = \frac{1}{4} \,.
\end{equation}

Thus, the lower quartile of $X$ can be found as,
\begin{align}
	F(LQ) = \frac{1}{3}(LQ-3) & = \frac{1}{4} \nn \\
			 \implies (LQ-3) & = 3 \times \frac{1}{4} \nn \\
			     \implies LQ & = \frac{3}{4} + 3 \nn \\
				      	     & = \frac{15}{4} \,.
\end{align}

\end{subsubquestions}

%------------------------------------------------------------------------------
% 3 b
%------------------------------------------------------------------------------
	
\subquestion

\begin{subsubquestions}
	
\subsubquestion

From \rdef{mod2:defn:ContinuousRV:CDF}, we can relate the probability density function, $f$, from the cumulative distribution function, $F$, via,
\begin{equation}
	f(x) = \ddd{}{x}F(x) \,.
\end{equation}

We need to calculate this in the different intervals over which $F$ is defined. The only non-trivial one occurs for $3\leq x \leq 6$,
\begin{align}
	f(x) & = \ddd{}{x}\left(\frac{1}{3}(x-3) \right) \nn \\
	     & = \frac{1}{3} \,.
\end{align}

Otherwise,
\begin{align}
	f &= \ddd{}{x} F = 0\,.
\end{align}

\rfig{2010:q3:fig:fGraph} shows $y=f(x)$.
\begin{figure}[H]
	\begin{center}
		\includegraphics{../2010/figures/2010q3-b-i}
		\caption{\label{2010:q3:fig:fGraph} Graph of $y=f(x)$.}
	\end{center}
\end{figure}

Summarizing,
\begin{align}
	f(x) =\begin{cases} 
	0, & x < 3 \\
	\frac{1}{3}, & 3 \leq x \leq 6 \\
	0, & x > 6 
		\end{cases}
\end{align}

%------------------------------------------------------------------------------

\subsubquestion

Using \rdef{mod2:defn:ContinuousRV:Expectation}, $E(X)$ is given as,
\begin{align}
	E(X) & = \int_{-\infty}^{\infty}x f(x)\dd x \,.
\end{align}

First, we must split up the intergral into the appropriate regions,
\begin{align}
	E(X)& = \int_{-\infty}^{3}x f(x)\,\dd x + \int_{3}^{6}x f(x)\,\dd x +\int_{6}^{\infty}x f(x)\,\dd x \,,
\end{align}
before we can evaluate,
\begin{align}
	E(X)& = 0+ \int_{3}^{6} \left(x \times \frac{1}{3} \right)\dd x+ 0\nn \\
	    & = \left[\frac{x^2}{6} \right]^{6}_{3} \nn \\
	    & = \left( \frac{6^2}{6} \right) - \left( \frac{3^2}{6} \right) \nn \\
	    & = 6 - \frac{9}{6} \nn \\
	    & = 4.5 \,.
\end{align}

%------------------------------------------------------------------------------

\subsubquestion

Using \rdef{mod2:defn:ContinuousRV:Variance}, we know that Var(X) is given by,
\begin{align}
	\var(X) & = E(X^2) - {E(X)}^2\, . \label{2010:q3:Var}
\end{align}

We then proceed by first finding $E(X^2)$,
\begin{align}
	E(X^2)& = \int_{-\infty}^{\infty}x^2 f(x) \, \dd x\,.
\end{align}

Again, the integral must be split up as,
\begin{align}
	E(X^2) &= \int_{-\infty}^3 f(x) \, \dd x + \int_{3}^6 f(x) \, \dd x + \int_6^\infty f(x) \, \dd x \,,
\end{align}
before we can substitute and evaluate as,
\begin{align}
E(X^2) & = 0+ \int_{3}^{6} \left( x^2 \times \frac{1}{3} \right) \, \dd x \nn + 0 \\
		 & = \left[\frac{x^3}{9} \right]^{6}_{3} \nn \\
		 & = \left( \frac{6^3}{9} \right) - \left( \frac{3^3}{9} \right) \nn \\
		 & = 24 -3=21 \,.
\end{align}

Finally, substituting our values into \req{2010:q3:Var}, we can determine $\var(X)$ as,
\begin{align}
\var(X) & = E(X^2) - {E(X)}^2 \nn \\
& = 21 - 4.5^2 \nn \\
& = \frac{3}{4} \,.
\end{align}

\end{subsubquestions}

\end{subquestions}