%------------------------------------------------------------------------------
% Author(s):
% Varaun Ramgoolie
% Copyright:
%  Copyright (C) 2020 Brad Bachu, Arjun Mohammed, Varaun Ramgoolie, Nicholas Sammy
%
%  This file is part of Applied-Mathematics-Unit2 and is distributed under the
%  terms of the MIT License. See the LICENSE file for details.
%
%  Description:
%     Year: 2010
%     Module: 1
%     Question: 2
%------------------------------------------------------------------------------

\begin{subquestions}

%------------------------------------------------------------------------------
% 1 a--------------------------------------------------------------------------
%------------------------------------------------------------------------------

\subquestion

Given that at least two members must vote yes for current to pass, we have to discard a simple $\boldsymbol{A \lor B \lor C}$ setup which will work even if \textbf{ONLY} one member votes yes.  Similarly, a simple $\boldsymbol{A \land B \land C}$ setup must be discarded because it will only work if \textbf{ALL} three members vote yes.  Hence, we can deduce that this circuit needs a combination of AND and OR setups.

Consider the three possible pairs of voters, $\boldsymbol{A \land B}$, $\boldsymbol{A \land C}$ and $\boldsymbol{B \land C}$. If at least two members answer yes, then at least one of these pairs will be switched on. Coversely, if one of these pairs are switched on, then at least two voters must have voted yes. Therefore we can construct the switching circuit of $\boldsymbol{(A \land B) \lor (A \land C) \lor (B \land C)}$ below.
\begin{center}
	
\begin{circuitikz}
	\draw [color=black, thin] (0,0) -- (2,0);
	\draw [color=black, thin] (2,0) -- (2,2);
	\draw [color=black, thin] (2,0) -- (2,-2);

	\draw (2,2) to[normal open switch, *-*](6,2);
	\draw (6,2) to[normal open switch, *-*](10,2);

	\path (2,2) -- (6,2) node[pos=0.5,below]{A};
	\path (6,2) -- (10,2) node[pos=0.5,below]{B};

	\draw (2,-2) to[normal open switch, *-*](6,-2);
	\draw (6,-2) to[normal open switch, *-*](10,-2);

	\path (2,-2) -- (6,-2) node[pos=0.5,below]{B};
	\path (6,-2) -- (10,-2) node[pos=0.5,below]{C};

	\draw (2,0) to[normal open switch, *-*](6,0);
	\draw (6,0) to[normal open switch, *-*](10,0);

	\path (2,0) -- (6,0) node[pos=0.5,below]{A};
	\path (6,0) -- (10,0) node[pos=0.5,below]{C};

	\draw [color=black, thin] (10,2) -- (10,0);
	\draw [color=black, thin] (10,-2) -- (10,0);

	\draw [color=black, thin] (10,0) -- (12,0);

\end{circuitikz}


 

\end{center}

%------------------------------------------------------------------------------
% 1 b--------------------------------------------------------------------------
%------------------------------------------------------------------------------

\subquestion

We can construct the following truth table to show that the proposition 
\begin{equation} 
	\boldsymbol{(\sim p ~\lor \sim q) \implies (p ~\land \sim q)}
 \end{equation} always takes the value of $\boldsymbol{p}$.

\begin{table}[h]
	\centering
	\begin{tabular}{|c|c|c|c|c|c|c|}
		\hline
		$\boldsymbol{p}$ & $\boldsymbol{q}$ & $\boldsymbol{\sim p}$ & $\boldsymbol{\sim q}$ & $\boldsymbol{(\sim p \ \lor \sim q)}$ & $\boldsymbol{(p \ \land \sim q)}$ & $\boldsymbol{(\sim p \ \lor \sim q) \implies (p \ \land \sim q)}$ \\
		\hline
		0 & 0 & 1 & 1 & 1 & 0 & 0 \\
		0 & 1 & 1 & 0 & 1 & 0 & 0 \\
		1 & 0 & 0 & 1 & 1 & 1 & 1 \\
		1 & 1 & 0 & 0 & 0 & 0 & 1 \\
		\hline
	\end{tabular}
	\caption{\label{2011:q2:tab:TruthTab1} Showing the truth values of $\boldsymbol{(\sim p ~\lor \sim q) \implies (p ~\land \sim q)}$}\,.
\end{table}

%------------------------------------------------------------------------------
% 1 c--------------------------------------------------------------------------
%------------------------------------------------------------------------------

\subquestion

We can draw a logic gate circuit for $\boldsymbol{a \implies b}$, using the given equivalence statement that $\boldsymbol{a \implies b \equiv \ \sim a \lor b}$.
\begin{figure}[H]
	\begin{center}
		\includegraphics{../2010/figures/2010q2Circuit1}
		\caption{\label{2010:q2:fig:Circuit1} Showing the logic gate equivalent of $\boldsymbol{\sim a \lor b}$.}
	\end{center}
\end{figure}

%------------------------------------------------------------------------------
% 1 d--------------------------------------------------------------------------
%------------------------------------------------------------------------------

\subquestion

\begin{subsubquestions}

%-----------------------------------------------------------------------------

\subsubquestion

We can complete the table given for the activity network as shown below.
\begin{table}[H]
	\centering
	\begin{tabular}{|c|c|c|c|}
		\hline
		Activity & Earliest Start Time & Latest Start Time & Float Time \\
		\hline
		A & 0 & 0 & 0 \\
		B & 6 & 6 & 0 \\
		C & 16 & 16 & 0 \\
		D & 6 & 17 & 11 \\
		E & 18 & 19 & 1 \\
		F & 18 & 18 & 0 \\
		G & 22 & 22 & 0 \\
		\hline
	\end{tabular}
	\caption{\label{2011:q2:tab:CritPath} Float time of the activities.}
\end{table}

%-----------------------------------------------------------------------------

\subsubquestion

From \rtab{2011:q2:tab:CritPath} and using \rdef{mod1:defn:FloatTime} and \rdef{mod1:defn:CritPath}, we can determine the critical path by looking at the path which has 0 float time:
\begin{equation}
	\text{Start} \rightarrow A \rightarrow B \rightarrow C \rightarrow F \rightarrow G \rightarrow \text{Finish}\,.
\end{equation}

\end{subsubquestions}

\end{subquestions}


