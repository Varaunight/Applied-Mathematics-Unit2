%------------------------------------------------------------------------------
% Author(s):
% Varaun Ramgoolie
% Copyright:
%  Copyright (C) 2020 Brad Bachu, Arjun Mohammed, Varaun Ramgoolie, Nicholas Sammy
%
%  This file is part of Applied-Mathematics-Unit2 and is distributed under the
%  terms of the MIT License. See the LICENSE file for details.
%
%  Description:
%     Year: 2012
%     Module: 2
%     Question: 4 
%------------------------------------------------------------------------------

%------------------------------------------------------------------------------
% 4 a
%------------------------------------------------------------------------------

\begin{subquestions}
	
\subquestion

There are 7 balls with numbers written on them. Two balls have the number 2 written on them, three balls have the number 3 on them, and 2 balls has the number 4 on them. In the system that we are given, the balls are drawn at random and they are not replaced. We have defined $X$ as the sum of the numbers drawn on two balls. 

\begin{subsubquestions}
	
\subsubquestion

We want to show that $P(X=8)=\frac{1}{21}$. For $X=8$, the first draw must be a 4 and the second draw must also be a 4. The probability that we get a 4 on the first draw is,
\begin{align}
	P(\text{first ball is 4}) & = \frac{\text{Number of ways to draw a 4}}{\text{Total number of ways to draw a ball}} \nn \\
	                          & = \frac{2}{7} \,.
\end{align}

Since there is no replacement, the probability that the second ball that we draw is also a 4 is,
\begin{align}
	P(\text{second ball is also 4}) & = \frac{\text{Number of ways to draw a 4}}{\text{Total number of ways to draw a ball}} \nn \\
	                                & = \frac{1}{6} \,.
\end{align}

Thus, we can calculate $P(X=8)$ as follows,
\begin{align}
	P(X=8) & = P(\text{first ball is 4}) \times P(\text{second ball is also 4}) \nn \\
	       & = \frac{2}{7} \times \frac{1}{6} \nn \\
	       & = \frac{1}{21} \,.
\end{align}

%------------------------------------------------------------------------------

\subsubquestion

The possible values of $X$ are 4, 5, 6, 7 and 8.

%------------------------------------------------------------------------------

\subsubquestion

In order to find the probabilities of all the values of $X$, we can either do calculations or we can create a Sample Space for $X$. See \rtab{2012:q4:Sample} for the sample space of $X$.

\begin{table}[ht]
	\centering
	\begin{tabular}{|c|c|c|c|c|c|c|c|}	
	\hline
	  & 1st "2" & 2nd "2" & 1st "3" & 2nd "3" & 3rd "3" & 1st "4" & 2nd "4" \\
	\hline 
	1st "2" & - & 4 & 5 & 5 & 5 & 6 & 6 \\
	\hline
	2nd "2" & 4 & - & 5 & 5 & 5 & 6 & 6 \\
	\hline
	1st "3" & 5 & 5 & - & 6 & 6 & 7 & 7 \\
	\hline
	2nd "3" & 5 & 5 & 6 & - & 6 & 7 & 7  \\
	\hline
	3rd "3" & 5 & 5 & 6 & 6 & - & 7 & 7 \\
	\hline
	1st "4" & 6 & 6 & 7 & 7 & 7 & - & 8  \\
	\hline
	2nd "4" & 6 & 6 & 7 & 7 & 7 & 8 & -  \\
	\hline	
	\end{tabular}
	\caption{\label{2012:q4:Sample} Sample Space of $X$.}
\end{table}

We can find the probabilities of all the values of $X$ as follows,
\begin{align}
	P(X=x) & = \frac{\text{Number of ways that x occurs in Sample Space}}{\text{Total outcomes in Sample Space}} \nn \\
	       & = \frac{\text{Number of ways that x occurs in Sample Space}}{42}\,.	
\end{align}

Thus, we get that,
\begin{align}
	P(X=4) & = \frac{2}{42} \nn \\
	       & = \frac{1}{21} \,. \\ \nn \\
	P(X=5) & = \frac{12}{42} \nn \\
	       & = \frac{2}{7} \,.	\\ \nn \\
	P(X=6) & = \frac{14}{42} \nn \\
	       & = \frac{1}{3} \,. \\ \nn \\
	P(X=7) & = \frac{12}{42} \nn \\
		   & = \frac{2}{7} \,.\\ \nn \\
	P(X=8) & = \frac{2}{42} \nn \\
		   & = \frac{1}{21} \,.	   
\end{align}

%------------------------------------------------------------------------------

\subsubquestion

From \rdef{mod2:defn:Discrete:Expectation}, we know that,
\begin{equation}
	E(X)= \sum_{\forall k} x_k P(X=x_k). 
\end{equation}

Thus, we can calculate $E(X)$ as follows,
\begin{align}
	E(X) & = \sum_{\forall k} x_k P(X=x_k) \nn \\
	     & = \left(4 \times P(X=4)\right) + \left(5 \times P(X=5)\right) + \left(6 \times P(X=6)\right) + \left(7 \times P(X=7)\right) + \left(8 \times P(X=8) \right) \nn \\
	     & = \left(4 \times \frac{1}{21}\right) + \left(5 \times \frac{2}{7}\right) + \left(6 \times \frac{1}{3}\right) + \left(7 \times \frac{2}{7}\right) + \left(8 \times \frac{1}{21} \right) \nn \\
	     & = \frac{4}{21} + \frac{10}{7} + \frac{6}{3} + \frac{14}{7} + \frac{8}{21} \nn \\
	     & = 6\,.
\end{align}

%------------------------------------------------------------------------------

\subsubquestion

We know that the Standard Deviation of $X$ is $\sqrt{\text{Variance}}$. From \rdef{mod2:defn:Discrete:Variance}, 
\begin{align}
	\text{Variance} & = E[X^2] - (E[X])^2 \nn \\
	           & = \sum_{\forall k} x_k^2 P(X=x_k) - (E(X))^2 \,.
\end{align}

We must first calculate $E(X^2)$ as follows,
\begin{align}
	\hspace{-100pt}
	E(X^2) & = \sum_{\forall k} x_k^2 P(X=x_k) \nn \\
	       & = \left(4^2 \times P(X=4)\right) + \left(5^2 \times P(X=5)\right) + \left(6^2 \times P(X=6)\right) + \left(7^2 \times P(X=7)\right) + \left(8^2 \times P(X=8) \right) \nn \\
	       & = \left(16 \times \frac{1}{21}\right) + \left(25 \times \frac{2}{7}\right) + \left(36 \times \frac{1}{3}\right) + \left(49 \times \frac{2}{7}\right) + \left(64 \times \frac{1}{21} \right) \nn \\
	       & = \frac{16}{21} + \frac{50}{7} + \frac{36}{3} + \frac{98}{7} + \frac{64}{21} \nn \\
	       & = \frac{776}{21} \,.
\end{align}

Thus, we can get that,
\begin{align}
	\text{Standard Deviation} & = \sqrt{\text{Variance}} \nn \\
	                          & = \sqrt{E[X^2] - (E[X])^2} \nn \\
	                          & = \sqrt{\frac{776}{21}- 6^2} \nn \\
	                          & = \sqrt{\frac{20}{21}} \,.
\end{align}

\end{subsubquestions}
	
%------------------------------------------------------------------------------
% 4 b
%------------------------------------------------------------------------------
	
\subquestion

We are given that, in some factory, 4\% of cameras produced are defective. Let $X$ be the number of defective cameras produced by the factory. This can be given as a Binomial Distribution with $p=0.04$, denoted as
\begin{equation}
	X \sim \text{Bin}(n,0.04) \,.
\end{equation}

\begin{subsubquestions}
	
\subsubquestion

We want to find the probability that, in a sample of 20 cameras, exactly 4 are defective. From \rdef{mod2:defn:Binomial}, we know that,
\begin{equation}
	P(X = x) = { 20 \choose x} \times 0.04^x \times (1-0.04)^{20-x} \,.
\end{equation}

Thus, we can calculate $P(X=4)$ as follows,
\begin{align}
	P(X = 4) & = { 20 \choose 4} \times 0.04^4 \times (1-0.04)^{20-4} \nn \\
	         & = 4845 \times 0.04^4 \times (0.96)^{16} \nn \\
	         & = 0.0065 \,.
\end{align}

%------------------------------------------------------------------------------

\subsubquestion

We want to find the probability that, in a sample of 100 cameras, at most 2 are defective. Since $X$ is discrete,
\begin{equation}
	P(X \leq 2) = P(X=0)+P(X=1)+P(X=2) \,.
\end{equation}

From \rdef{mod2:defn:Binomial}, we know that,
\begin{equation}
	P(X = x) = { 100 \choose x} \times 0.04^x \times (1-0.04)^{100-x} \,.
\end{equation}

Thus, we can calculate $P(X=4)$ as follows,
\begin{align}
	P(X = 0) & = { 100 \choose 0} \times 0.04^0 \times (1-0.04)^{100-0} \nn \\
             & = 1 \times 0.04^0 \times (0.96)^{100} \nn \\
	         & = 0.0169 \,. \\ \nn \\
	P(X = 1) & = { 100 \choose 1} \times 0.04^1 \times (1-0.04)^{100-1} \nn \\
			 & = 100 \times 0.04^1 \times (0.96)^{99} \nn \\
			 & = 0.0703 \,. \\ \nn \\	 
	P(X = 2) & = { 100 \choose 2} \times 0.04^2 \times (1-0.04)^{100-2} \nn \\
			 & = 4950 \times 0.04^2 \times (0.96)^{98} \nn \\
			 & = 0.1450 \,. \\ \nn \\
	P(X \leq 2) & = P(X=0)+P(X=1)+P(X=2) \nn \\
	            & = 0.0169+0.0703+0.1450 \nn \\
	            & = 0.2322 \,.		 
\end{align}

\end{subsubquestions}

\end{subquestions}