%------------------------------------------------------------------------------
% Author(s):
% Varaun Ramgoolie
% Copyright:
%  Copyright (C) 2020 Brad Bachu, Arjun Mohammed, Varaun Ramgoolie, Nicholas Sammy
%
%  This file is part of Applied-Mathematics-Unit2 and is distributed under the
%  terms of the MIT License. See the LICENSE file for details.
%
%  Description:
%     Year: 2005 C
%     Module: 2
%     Question: 4
%------------------------------------------------------------------------------

%------------------------------------------------------------------------------
% 4 a
%------------------------------------------------------------------------------

\begin{subquestions}
	
\subquestion

\begin{subsubquestions}
	
\subsubquestion

Let $X$ be the discrete random variable representing ``the number of adults dissatisfied with their health coverage from a group of 15.'' We determine that
\begin{equation}
	X \sim \text{Bin}(15, 0.05) \,.
\end{equation}

\begin{subsubsubquestions}
	
\subsubsubquestion
From \rdef{mod2:defn:Binomial}, we know that,
\begin{equation}
	P(X = x) = { 15 \choose x} (0.05)^x (1-0.05)^{15-x} \,. \label{2005:q4:Bin1}
\end{equation}
Hence,
\begin{align}
	P(X=2) & = { 15 \choose 2} \times (0.05)^2 \times (1-0.05)^{15-2} \nn \\
	       & = 105 \times (0.05)^2 \times (0.95)^{13} \nn \\
	       & = 0.135 \,.
\end{align}

%------------------------------------------------------------------------------

\subsubsubquestion
Since $X$ is discrete, we know that,
\begin{align}
	P(X>2) &= 1 - P(X\leq2) \nn \\
				&= 1 - (P(X=0)+P(X=1)+P(X=2)) \,. \label{2005:q4:DisBin}
\end{align}

Using \req{2005:q4:Bin1}, we can calculate the individual probabilities,
\begin{align}
	P(X=0) & = { 15 \choose 0} \times (0.05)^0 \times (0.95)^{15-0} \nn \\
	       & = 1 \times 1 \times (0.95)^{15} \nn \\
	       & = 0.463 \,. \\ 
	P(X=1) & = { 15 \choose 1} \times (0.05)^1 \times (0.95)^{15-1} \nn \\
	       & = 15 \times 0.05 \times (0.95)^{14} \nn \\
	       & = 0.366 \,. \\
	P(X=2) & = { 15 \choose 2} \times (0.05)^2 \times (0.95)^{15-2} \nn \\
	       & = 15 \times 0.05 \times (0.95)^{13} \nn \\
	       & = 0.135 \,. 
\end{align}

Substituting these values into \req{2005:q4:DisBin}, we can find that,       
\begin{align}
	P(X>2) & = 1 - (P(X=0)+P(X=1)+P(X=2)) \nn \\
	       & = 1 - (0.463+0.366+0.135) \nn \\
	       & = 1 - 0.964 \nn \\
	       & = 0.036 \,. 
\end{align}

\end{subsubsubquestions}

%------------------------------------------------------------------------------

\subsubquestion

Let $X$ be the discrete random variable representing `` the number of adults dissatisfied with their health coverage from a group of 60''. Given that $n=60$ and $p=0.05$, at first we can say
\begin{align}
X\sim \text{Bin}(60,0.05) \,.
\end{align}

However, since
\begin{align}
	n=60 & >30, \text{and} \nn \\
	np = 60 & \times 0.05 = 3 < 5 \,,
\end{align}
we can apply the Poisson approximation to the Binomial distribution, \rdef{mod2:defn:PoissonApproxBinomial}, as follows,
\begin{equation}
	X \sim \text{Pois}(np) \equiv X \sim \text{Pois}(3) \,.
\end{equation}

Now we can use this to compute $P(X\leq3)$. Since $X$ is discrete, 
\begin{equation}
	P(X \leq 3) = P(X=0)+P(X=1)+P(X=2)+P(X=3) \,. \label{2005:q4:DisPois}
\end{equation}

For a Poisson random variable, we know that according to \rdef{mod2:defn:Poisson}, 
\begin{equation}
	P(X = x) =\frac{ 3 ^ x \times e^{-3}}{x!} \,. \label{2005:q4:Pois1}
\end{equation}


From \req{2005:q4:Pois1}, we need to calculate the following,
\begin{align}
	P(X = 0) & = \frac{ 3 ^ 0 \times e^{-3}}{0!} \nn \\
	         & = \frac{ 1 \times e^{-3}}{1} \nn \\
	         & = e^{-3} \,. 
\end{align}
\begin{align}
	P(X = 1) & = \frac{ 3 ^ 1 \times e^{-3}}{1!} \nn \\
			 & = \frac{ 3 \times e^{-3}}{1} \nn \\
			 & = 3e^{-3} \,. 
\end{align}
\begin{align}
	P(X = 2) & = \frac{ 3 ^ 2 \times e^{-3}}{2!} \nn \\
			 & = \frac{ 9 \times e^{-3}}{2} \nn \\
			 & = \frac{9e^{-3}}{2} \,.
\end{align}
\begin{align}
	P(X = 3) & = \frac{ 3 ^ 3 \times e^{-3}}{3!} \nn \\
		  	 & = \frac{ 27 \times e^{-3}}{6} \nn \\
			 & = \frac{9e^{-3}}{2} \,. 
\end{align}

Substituting these values into \req{2005:q4:DisPois}, we find,
\begin{align}
	P(X \leq 3) & = P(X=0)+P(X=1)+P(X=2)+P(X=3) \nn \\
				& = e^{-3} \times \left(1+3+\frac{9}{2}+\frac{9}{2}\right) \nn \\
	            & = e^{-3} \times 13 \nn \\
	            & = 0.647 \,.
\end{align}

\end{subsubquestions}
	
%------------------------------------------------------------------------------
% 4 b
%------------------------------------------------------------------------------

\subquestion

\begin{subsubquestions}
	
\subsubquestion
From \rdef{mod2:defn:Conditional}, we know that the conditional probability of $A$ given $B$ is,
\begin{align}
	P(A|B) & = \frac{P(A\cap B)}{P(B)} \,.
\end{align}

Hence, we need to determine $P(A \cap B)$. Using \rprop{mod2:prop:SetPropositions},
\begin{align}
	P(A \cup B) & = P(A) + P(B) - P(A \cap B) \nn \\
	        0.9 & = 0.5+0.6-P(A \cap B) \nn \\
	        \implies P(A \cap B) & = 0.5+0.6-0.9 \nn \\
	                             & = 0.2 \,.
\end{align}
	
Now, we can substitute in the above to find
\begin{align}
	P(A|B) & = \frac{P(A\cap B)}{P(B)} \nn \\
	       & = \frac{0.2}{0.6} \nn \\
	       & = \frac{1}{3} \,.
\end{align}

%------------------------------------------------------------------------------

\subsubquestion

See \rfig{2005:q4:fig:Venn}.

\begin{figure} [H]
	\begin{center}
		\includegraphics{../2005/figures/2005q4Venn}
		\caption{\label{2005:q4:fig:Venn} Venn Diagram.}
	\end{center}
\end{figure}

\end{subsubquestions}

\end{subquestions}