%------------------------------------------------------------------------------
% Author(s):
% Varaun Ramgoolie
% Copyright:
%  Copyright (C) 2020 Brad Bachu, Arjun Mohammed, Varaun Ramgoolie, Nicholas Sammy
%
%  This file is part of Applied-Mathematics-Unit2 and is distributed under the
%  terms of the MIT License. See the LICENSE file for details.
%
%  Description:
%     Year: 2005 C
%     Module: 1
%     Question: 2 
%------------------------------------------------------------------------------

\begin{subquestions}
	
%------------------------------------------------------------------------------
% 2 a -------------------------------------------------------------------------
%------------------------------------------------------------------------------

\subquestion

The objective function can be given as,
\begin{equation}
	P = 300x + 550y \,.
\end{equation}

%------------------------------------------------------------------------------
% 2 b -------------------------------------------------------------------------
%------------------------------------------------------------------------------

\subquestion

The 4 constraints to this problem are, 
\begin{align}
	\text{Constraint of number of pine cabinets:} & \ x \geq 0 \,, \nn \\
	\text{Constraint of number of mahogany cabinets:} & \ y \geq 0 \,, \nn \\
	\text{Constraint of time:} & \ 3x + 5y \leq 30 \,, \nn \\
	\text{Constraint of cost:} & \ 100x + 250y \leq 1250 \,.
\end{align}

As it is required to give each constraint in simplest form, the Cost Constraint can be reduced to:
\begin{equation}
	100x + 250y \leq 1250 \implies 2x + 5y \leq 25 \,.
\end{equation}

%------------------------------------------------------------------------------
% 2 c -------------------------------------------------------------------------
%------------------------------------------------------------------------------

\subquestion

The feasible region is shaded in \rfig{2005:q2:fig:LinGraph}.
\begin{figure}[h]
	\begin{center}
		\includegraphics{../2005/figures/2005q2LinGraph}
		\caption{\label{2005:q2:fig:LinGraph} Linear Programming Graph.}
	\end{center}
\end{figure}

%------------------------------------------------------------------------------
% 2 d -------------------------------------------------------------------------
%------------------------------------------------------------------------------
\subquestion

By performing a Tour of the Vertices (\rdef{mod1:defn:TourOfVertices}), we can evaluate the objective Profit function.

\begin{table}[H]
	\centering
	\begin{tabular}{|c|c|}
		\hline
		Vertice & $P = 300x + 550y$ \\
		\hline
		(0, 0) & 0 \\
		(0, 5) & 2750 \\
		(5, 3) & 2550 \\
		(10, 0) & 3000 \\
		\hline
	\end{tabular}
	\caption{\label{2005:q2:tab:Profit} Tour of Vertices}
\end{table}

From \rtab{2005:q2:tab:Profit}, we see that maximum profit $P$ is $ \$ 3000$ when $x = 10$ and $y = 0$.

%------------------------------------------------------------------------------
% 2 e -------------------------------------------------------------------------
%------------------------------------------------------------------------------

\subquestion

The Profit Line method can also be used to determine the maximum value of $P$. It relies on drawing parallel lines, starting from the origin, which have the gradient of the objective function. This line is then drawn at different points on the $x$ or $y$ axes. The furthest point on the feasible region which the Profit Line touches (just before leaving the feasible region) is the point which will yield the maximum value of the objective function. 

\end{subquestions}

