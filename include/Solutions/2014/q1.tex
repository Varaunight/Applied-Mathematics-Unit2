%------------------------------------------------------------------------------
% Author(s):
% Varaun Ramgoolie
% Copyright:
%  Copyright (C) 2020 Brad Bachu, Arjun Mohammed, Varaun Ramgoolie, Nicholas Sammy
%
%  This file is part of Applied-Mathematics-Unit2 and is distributed under the
%  terms of the MIT License. See the LICENSE file for details.
%
%  Description:
%     Year: 2014
%     Module: 1
%     Question: 1
%------------------------------------------------------------------------------

\begin{subquestions}
	
%------------------------------------------------------------------------------
% 1 a--------------------------------------------------------------------------
%------------------------------------------------------------------------------

\subquestion

\begin{equation}
	q \implies \sim p\,. 
\end{equation}

%------------------------------------------------------------------------------
% 1 b--------------------------------------------------------------------------
%------------------------------------------------------------------------------

\subquestion

See \rtab{2014:q1:tab:Tab1}. Note that the inverse of ($p \implies \sim q$) is ($\sim p \implies q$).
\begin{table}[ht]
	\centering
	\begin{tabular}{|c|c|c|c|c|}
		\hline
		p & $\sim$ q & $\sim$ p & q & ($\sim$ p $\implies$ q) \\
		\hline
		0 & 0 & 1 & 1 & 1 \\
		0 & 1 & 1 & 0 & 0 \\
		1 & 0 & 0 & 1 & 1 \\
		1 & 1 & 0 & 0 & 1 \\
		\hline
	\end{tabular}
	\caption{\label{2014:q1:tab:Tab1} Truth Table of $\sim p \implies q$.}
\end{table}
		
%------------------------------------------------------------------------------
% 1 c--------------------------------------------------------------------------
%------------------------------------------------------------------------------
		
\subquestion

\begin{subsubquestions}
	
\subsubquestion

See \rtab{2014:q1:tab:Tab2}.
\begin{table}[ht]
	\centering
	\begin{tabular}{|c|c|c|c|c|c|}
		\hline
		p & q & r & (p $\implies$ q) & (q $\implies$ r) & (p $\implies$ q) $\lor$ (q $\implies$ r) \\
		\hline
		0 & 0 & 0 & 1 & 1 & 1 \\
		0 & 0 & 1 & 1 & 1 & 1 \\
		0 & 1 & 0 & 1 & 0 & 1 \\
		0 & 1 & 1 & 1 & 1 & 1 \\
		1 & 0 & 0 & 0 & 1 & 1 \\
		1 & 0 & 1 & 0 & 1 & 1 \\
		1 & 1 & 0 & 1 & 0 & 1 \\
		1 & 1 & 1 & 1 & 1 & 1 \\
		\hline
	\end{tabular}
	\caption{\label{2014:q1:tab:Tab2} Truth Table of $(p \implies q) \lor (q \implies r)$.}
\end{table}

%-----------------------------------------------------------------------------

\subsubquestion

From \rdef{mod1:defn:Tautology}, we can see that $(p \implies q) \lor (q \implies r)$ is a tautology.
\end{subsubquestions}

%------------------------------------------------------------------------------
% 1 d--------------------------------------------------------------------------
%------------------------------------------------------------------------------

\subquestion

The Boolean expression for the circuit shown is,
\begin{equation}
	\sim (a \land b) \lor \sim c\,.
\end{equation}

%------------------------------------------------------------------------------
% 1 e--------------------------------------------------------------------------
%------------------------------------------------------------------------------

\subquestion

\begin{subsubquestions}

\subsubquestion

The switching circuit for $A \lor (B \land C)$ is shown below.

\begin{circuitikz}
	\draw [color=black, thick] (0,0) -- (2,0);
	\draw [color=black, thick] (2,0) -- (2,1);
	\draw [color=black, thick] (2,0) -- (2,-1);
	\draw (2,1) to[normal open switch, *-*](6,1);
	\path (2,1) -- (6,1) node[pos=0.5,below]{A};
	\draw (2,-1) to[normal open switch, *-*](4,-1);
	\path (2,-1) -- (4,-1) node[pos=0.5,below]{B};
	\draw (4,-1) to[normal open switch, *-*](6,-1);
	\path (4,-1) -- (6,-1) node[pos=0.5,below]{C};
	\draw [color=black, thick] (6,1) -- (6,0);
	\draw [color=black, thick] (6,-1) -- (6,0);
	\draw [color=black, thick] (6,0) -- (8,0);
\end{circuitikz}

%-----------------------------------------------------------------------------

\subsubquestion

Using \rdef{mod1:law:Distributive}, we see that,
\begin{equation}
	A \lor (B \land C) \equiv (A \lor B) \land (A \lor C) \,.
\end{equation}

\end{subsubquestions}

\end{subquestions}

