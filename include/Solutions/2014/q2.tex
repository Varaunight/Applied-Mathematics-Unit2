%------------------------------------------------------------------------------
% Author(s):
% Varaun Ramgoolie
% Copyright:
%  Copyright (C) 2020 Brad Bachu, Arjun Mohammed, Varaun Ramgoolie, Nicholas Sammy
%
%  This file is part of Applied-Mathematics-Unit2 and is distributed under the
%  terms of the MIT License. See the LICENSE file for details.
%
%  Description:
%     Year: 2014
%     Module: 1
%     Question: 2
%------------------------------------------------------------------------------

\begin{subquestions}

%-----------------------------------------------------------------------------
% 2 a-------------------------------------------------------------------------
%-----------------------------------------------------------------------------

\subquestion

\begin{subsubquestions}
	
\subsubquestion

We can construct the following activity network from the given activities.
\begin{figure}[H]
	\begin{center}
		\includegraphics{../2014/figures/ActNet}
		\caption{\label{2014:q2:fig:ActNet} Activity Network of the operation.}
	\end{center}
\end{figure}

%-----------------------------------------------------------------------------

\subsubquestion

We can complete the table as follows:
\begin{table}[H]
	\centering
	\begin{tabular}{|c|c|c|c|}
		\hline
		Activity&Earliest Start Time&Latest Start Time&Float Time\\
		\hline
		A & 0 & 0 & 0 \\
		B & 6 & 16 & 10 \\
		C & 6 & 13 & 7 \\
		D & 6 & 6 & 0 \\
		E & 9 & 9 & 0 \\
		F & 12 & 12 & 0 \\
		G & 12 & 12 & 0 \\
		H & 21 & 21 & 0  \\
		\hline
		
	\end{tabular}
\caption{\label{2014:tab:Act} Float Times of the activities.}
\end{table}

%-----------------------------------------------------------------------------

\subsubquestion

Using Def. \ref{mod1:defn:CritPath} and looking at Fig. \ref{2014:q2:fig:ActNet} and Table \ref{2014:tab:Act}, we can see that the two Critical Paths of this Activity Network are:
\begin{align}
	\text{Start} \rightarrow A \rightarrow D \rightarrow E \rightarrow F \rightarrow H \rightarrow \text{End} \nn \\
	\text{Start} \rightarrow A \rightarrow D \rightarrow E \rightarrow G \rightarrow H \rightarrow \text{End} \nn \\ \nn
\end{align}
	
\end{subsubquestions}

%-----------------------------------------------------------------------------
% 2 b-------------------------------------------------------------------------
%-----------------------------------------------------------------------------

\subquestion

\begin{subsubquestions}
	
\subsubquestion

We can write the following Boolean expression for the given switching circuit:
\begin{align}
	\boldsymbol{p \wedge (p \: \vee \sim q)} \label{2014:q2:eqn:Bool}
\end{align}

%-----------------------------------------------------------------------------

\subsubquestion

We can construct the truth table for the proposition $\boldsymbol{p \land (p ~\lor \sim q)}$
\begin{table}[ht]
	\centering
	\begin{tabular}{|c|c|c|c|c|}
		\hline
		$\boldsymbol{p}$ &$\boldsymbol{q}$ & $\boldsymbol{\sim q}$ & $\boldsymbol{p \ \vee \sim q}$ & $\boldsymbol{p \wedge ( p \ \vee \sim q)}$ \\
		\hline
		0 & 0 & 1 & 1 & 0 \\
		0 & 1 & 0 & 0 & 0 \\
		1 & 0 & 1 & 1 & 1 \\
		1 & 1 & 0 & 1 & 1 \\
		\hline
	\end{tabular}
	\caption{\label{2014:tab:TrthTab}Truth Table of $\boldsymbol{p \land (p ~\lor \sim q)}$.}
\end{table}

Note: Although the question did not ask for it, this truth table is a proof of the Absorptive Law (\rdef{mod1:law:Absorptive}).

\end{subsubquestions}

\end{subquestions}
