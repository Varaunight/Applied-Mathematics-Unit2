%------------------------------------------------------------------------------
% Author(s):
% Varaun Ramgoolie
% Copyright:
%  Copyright (C) 2020 Brad Bachu, Arjun Mohammed, Varaun Ramgoolie, Nicholas Sammy
%
%  This file is part of Applied-Mathematics-Unit2 and is distributed under the
%  terms of the MIT License. See the LICENSE file for details.
%
%  Description:
%     Year: 2008 May
%     Module: 1
%     Question: 1 
%------------------------------------------------------------------------------

\begin{subquestions}
	
%------------------------------------------------------------------------------
% 1 a--------------------------------------------------------------------------
%------------------------------------------------------------------------------

\subquestion

Let $x$ be the number of newspaper advertisements and let $y$ be the number of television advertisements.	
The objective function that we want to maximize is,
\begin{equation}
	C =2x+5y \,.
\end{equation} 

The constraints of this linear programming model are as follows,
\begin{align}
	1500x + 5000y & \leq 5000 \,, \nn \\
	1500x & \leq 30000 \,, \nn \\
	5000y & \geq 25000 \,, \nn \\
	x & \leq 2y \,.
\end{align}

%------------------------------------------------------------------------------
% 1 b--------------------------------------------------------------------------
%------------------------------------------------------------------------------

\subquestion

\begin{subsubquestions}

%-----------------------------------------------------------------------------
	
\subsubquestion

The feasible region is shaded in \rfig{2008M:q1:fig:Graph}.
\begin{figure}
	\begin{center}
		\includegraphics{../2007/figures/2008Mq1Graph}
		\caption{\label{2008M:q1:fig:Graph} Linear Programming Graph.}
	\end{center}
\end{figure}

%-----------------------------------------------------------------------------------------

\subsubquestion

Using \rdef{mod1:defn:TourOfVertices},
\begin{align}
	\text{Using (2,2)} \,, \nn \\
	P & = 2x + 7y \,, \nn \\
	& = (2 \times 2) + (7 \times 2) \,, \nn \\
	& = 18 \,. \\
	\text{Using (2,3)} \,, \nn \\
	P & = 2x + 7y \,, \nn \\
	& = (2 \times 2) + (7 \times 3) \,, \nn \\
	& = 25 \,. \\  \label{2012:q1:eqn:Profit}	  
	\text{Using (4,1)} \,, \nn \\
	P & = 2x + 7y \,, \nn \\
	& = (2 \times 4) + (7 \times 1) \,, \nn \\
	& = 15 \,.  \\
	\text{Using (2.4, 1)} \,, \nn \\
	P & = 2x + 7y \,, \nn \\
	& = (2 \times 2.4) + (7 \times 1) \,, \nn \\
	& = 11.8 \,. 
\end{align}

Thus, the maximum value of $P$ is $25$.

\end{subsubquestions}

\end{subquestions}

