%------------------------------------------------------------------------------
% Author(s):
% Varaun Ramgoolie
% Copyright:
%  Copyright (C) 2020 Brad Bachu, Arjun Mohammed, Varaun Ramgoolie, Nicholas Sammy
%
%  This file is part of Applied-Mathematics-Unit2 and is distributed under the
%  terms of the MIT License. See the LICENSE file for details.
%
%  Description:
%     Year: 2008 May
%     Module: 2
%     Question: 3 
%------------------------------------------------------------------------------

%------------------------------------------------------------------------------
% 3 a
%------------------------------------------------------------------------------

\begin{subquestions}
	
\subquestion

Let $X$ be the discrete random variable that represents ``the number of 3's thrown when a dice is rolled 4 times.'' 
\begin{equation}
	X \sim \text{Bin}\left(4, \frac{1}{6}\right) \,.
\end{equation}

From \rdef{mod2:defn:Binomial}, the probabilities of a Binomial distribution are,
\begin{equation}
	P(X = x) = { 4 \choose x} \times \left(\frac{1}{6} \right)^x \times \left(1-\frac{1}{6} \right)^{4-x} \,. \label{2008M:q3:BinEqn1}
\end{equation}
	
Using \req{2008M:q3:BinEqn1}, we can then calculate the expected probabilities, assuming a Binomial distribution,
\begin{align}
	P(X=0) & = { 4 \choose 0} \times \left(\frac{1}{6} \right)^0 \times \left(1-\frac{1}{6} \right)^{4-0} \nn \\
	       & = 1 \times 1 \times \left(\frac{5}{6} \right)^{4} \nn \\
	       & = \frac{625}{1296} \,,
\end{align}
\begin{align}
	P(X=1) & = { 4 \choose 1} \times \left(\frac{1}{6} \right)^1 \times \left(1-\frac{1}{6} \right)^{4-1} \nn \\
		   & = 4 \times \frac{1}{6} \times \left(\frac{5}{6} \right)^{3} \nn \\
		   & = \frac{500}{1296} \,,
\end{align}
\begin{align}
	P(X=2) & = { 4 \choose 2} \times \left(\frac{1}{6} \right)^2 \times \left(1-\frac{1}{6} \right)^{4-2} \nn \\
		   & = 6 \times \left(\frac{1}{6}\right)^2 \times \left(\frac{5}{6} \right)^{2} \nn \\
		   & = \frac{150}{1296} \,,
\end{align}
\begin{align}
	P(X=3) & = { 4 \choose 3} \times \left(\frac{1}{6} \right)^3 \times \left(1-\frac{1}{6} \right)^{4-3} \nn \\
		   & = 4 \times \left(\frac{1}{6}\right)^3 \times \left(\frac{5}{6} \right) \nn \\
      	   & = \frac{20}{1296} \,,
\end{align}
and,
\begin{align}
    P(X=4) & = { 4 \choose 4} \times \left(\frac{1}{6} \right)^4 \times \left(1-\frac{1}{6} \right)^{4-4} \nn \\
    	   & = 1 \times \left(\frac{1}{6}\right)^4 \times 1 \nn \\
    	   & = \frac{1}{1296} \,.
\end{align}

Now that we have our expected probailities, we can compute the expected frequencies. The results are summarized in \rtab{2008M:q3:Bin1}.
	
\begin{table}[H]
	\centering
	\begin{tabular}{|c|c|c|}
		\hline
		$X=x$ & $P(X=x)$ & Expected frequency, Total $\times P(X=x)$ \\
		\hline
		0 & 0.482 & 48.2 \\
		1 & 0.386 & 38.6 \\
		2 & 0.116 & 11.6 \\
		3 & 0.015 & 1.5 \\
		4 & 0.001 & 0.1 \\
		\hline
	\end{tabular}
	\caption{\label{2008M:q3:Bin1} Expected Frequencies.}
\end{table}

%------------------------------------------------------------------------------
% 3 b
%------------------------------------------------------------------------------

\subquestion

We are asked to perform a $\chi^2$ goodness-of-fit test at the 5\% significance level. From Section(CHI Squared notes), we know that,
\begin{equation}
	\chi^2_\text{calc.} = \sum_{\forall k} \left(\frac{(O_k-E_k)^2}{E_k}  \right) \,.
\end{equation}

Our hypotheses are,
\begin{itemize}
	\item $H_0$: The data follows a binomial distribution with parameters $n=4$ and $p=\frac{1}{6}$. 
	\item $H_1$: The data does not follow a binomial distribution with parameters $n=4$ and $p=\frac{1}{6}$.
\end{itemize}

We should note that the expected frequencies for $X=3$ and $X=4$ are both less than 5. Therefore, we must combine both classes with $X=2$ as follows,
\begin{table}[H]
	\centering
	\begin{tabular}{|c|c|c|c|}
		\hline 
		$X=x$ & Observed Frequency, O & Expected Frequency, E & $\frac{(O-E)^2}{E}$ \\
		\hline
		0 & 57 & 48.2 & 1.607 \\
		1 & 30 & 38.6 & 1.916 \\
		$\geq$ 2 & 13 & 13.2 & 0.003 \\
		\hline
		Total, $\chi^2_\text{calc.}$ & & & 3.526 \\
		\hline
	\end{tabular}
	\caption{\label{2008M:q3:ChiTab} Expected and Observed Frequencies of experiment.}
\end{table}

From \rtab{2008M:q3:ChiTab}, we see that our number of degrees of freedom, $\nu$, is,
\begin{equation}
	\nu = 3 - 1 = 2 \,.
\end{equation}

We reject $H_0$ if and only if,
\begin{align}
	\chi^2_\text{calc.} & > \chi^2_{\alpha}(\nu) \nn \\
	\implies \chi^2_\text{calc.} & > \chi^2_{0.05}(2) \nn \\
	\implies \chi^2_\text{calc.} & > 5.991 \,.
\end{align}

Since our value of $\chi^2_\text{calc.}=3.526$ is not greater than $\chi^2_{0.05}(2)=5.991$, we fail to reject $H_0$.

\end{subquestions}
