%------------------------------------------------------------------------------
% Author(s):
% Varaun Ramgoolie
%
% Copyright:
%  Copyright (C) 2020 Brad Bachu, Arjun Mohammed, Varaun Ramgoolie, Nicholas Sammy
%
%  This file is part of Applied-Mathematics-Unit2 and is distributed under the
%  terms of the MIT License. See the LICENSE file for details.
%
%  Description:
%     Year: 2008 May
%     Module: 3
%     Question: 6
%------------------------------------------------------------------------------

%------------------------------------------------------------------------------
% 6 a
%------------------------------------------------------------------------------

\begin{subquestions}
	
\subquestion

\begin{subsubquestions}
	
\subsubquestion

Newton's Second Law states that, the rate of change of linear momentum of an object is directly proportional to the Resultant Force which acts on the object. We can represent this mathematically as,
\begin{align} 
	\text{Resultant Force}, \vec{F} & \propto \ddd{\vec{p}}{t} \nn \\
	             & \propto \frac{\Delta \vec{p}}{\Delta t} \nn \\
	             & \propto \frac{m\vec{v}-m\vec{u}}{\Delta t} \nn \\
	             & \propto m\frac{\vec{v}-\vec{u}}{\Delta t} \nn \\
	             & = m\vec{a} \,.
\end{align}

It should be noted that $\vec{F}$ and $\vec{a}$ are vector quantities. \footnote{This means that the equality holds in their respective directions.}
\end{subsubquestions}
	
%------------------------------------------------------------------------------

\subsubquestion
	
\textbf{\textit{Sketch and Translate:}}
\addimage{../2007/figures/2008Mq6-Diagram1}{2008M:q6:Diagram1}{Forces acting on the particle.}
It is given that a particle is thrown up in the air with an initial speed, $u$ and varying resistive forces act against it. We should notice that the resistive force will be directed in the opposite direction of the motion of the body. At the greatest height, the velocity $v$ must be 0.\\




\textbf{\textit{Simplify and Diagram:}} \\
We can consider the rightmost diagram in \rfig{2008M:q6:Diagram1}. We want to find an expression for the greatest height reached by the particle. From Newton's 2nd Law, we know that the particle must have an acceleration in the $-y$ direction. Since we know that the resultant force (and by extension, the acceleration) is dependent on the speed of the particle, $v$, in the $+y$ direction, we can formulate an expression for the height of the particle and solve for the greatest height.

We will represent the movement of the particle in the $+y$ direction as $s_y$.
We should also assume that no other forces act on the particle except for those in \rfig{2008M:q6:Diagram1}. \\




\textbf{\textit{Represent Mathematically:}} \\
As there are no forces in the $x$ direction, we only need to consider the $y$ direction. From Newton's Second Law, we see that,
\begin{align}
	\sum F & = ma \nn \\
	-(mg+mkv^2)  & = ma  \nn \\
	\implies a  & = (-g -kv^2) \,.
\end{align}

As we can see, the acceleration is dependent on $v^2$ and is in the $-y$ direction (due to the minus sign). We can formulate a differential equation as follows,
\begin{align}
	a & = -g -kv^2 \nn \\
	v \ddd{v}{s_y} & = -g - kv^2 \nn \\
\end{align}

Separating these variables, we get,
\begin{align}
	\dd s_y & = \frac{-v}{g+kv^2}\dd v \nn \\
	\int_{y_1}^{y_2} \dd s_y & = \int_{v_1}^{v_2} \frac{-v}{g+kv^2}\dd v \,.
\end{align}




\textbf{\textit{Solve and Evaluate:}} \\
From our given information, we know that when $s_y=0$, $v=u$, and when $s_y=y_{\text{max}}$, $v=0$. Using these as our bounds of integration, we get that,
\begin{align}
	\int_{0}^{y_{\text{max}}} \dd s_y & = \int_{u}^{0} \frac{-v}{g+kv^2}\dd v \nn \\
	\int_{0}^{y_{\text{max}}} \dd s_y & = \frac{-1}{2k}\int_{u}^{0} \frac{2kv}{g+kv^2}\dd v \nn \\
	\left[s_y\right]_0^{y_{\text{max}}} & = \frac{-1}{2k} \left[\ln(g+kv^2)\right]_u^0 \nn \\
	y_{\text{max}} & = \frac{-1}{2k} \left[\ln(g)-\ln(g+ku^2)\right] \nn \\
	               & = \frac{1}{2k} \left[\ln(g+ku^2)-\ln(g)\right] \nn \\
	               & = \frac{1}{2k} \left(\ln\left(\frac{g+ku^2}{g}\right)\right) \nn \\
	 			   & = \frac{1}{2k} \left(\ln\left(1 + \frac{ku^2}{g}\right)\right) \,.
\end{align} \footnote{In our second line, we use a trick to integrate this function. By multiplying by $\frac{2k}{2k}$, we converted our integral into the form of $\int \frac{f'(x)}{f(x)}$ which is equal to $\ln(f(x))$.} \\

%------------------------------------------------------------------------------
% 6 b
%------------------------------------------------------------------------------

\subquestion

\textbf{\textit{Sketch and Translate:}} \\
\addimage{../2007/figures/2008Mq6-Sketch2}{2008M:q6:Sketch2}{Forces along trapezoid}
We are given a series of forces, which act along the sides of a trapezium ABCD, that are in equilibrium.



\textbf{\textit{Simplify and Diagram:}} \\ 
\addimage{../2007/figures/2008Mq6-Diagram2}{2008M:q6:Diagram2}{Forces in equilibrium, acting on a point particle.}
As we are given that the system is in equilibrium, we can collapse the trapezoid and assume that all the forces act on a single point particle.

From \rfig{2008M:q6:Sketch2}. let,
\begin{itemize}
	\item $\vec{AB}$ be the force acting from A to B.
	\item $\vec{BC}$ be the force acting from B to C.
	\item $\vec{CD}$ be the force acting from C to D.
	\item $\vec{DA}$ be the force acting from D to A.
\end{itemize}

Due to equilibrium, the resultant force on system must be 0. By resolving the forces in the $x$ and $y$ directions, we can solve for $p$ and $q$.\\




\textbf{\textit{Represent Mathematically:}} \\
Resolving $\vec{AB}$, we get,
\begin{align}
	\vec{AB} & = |\vec{AB}|\xhat \nn \\
	         & = 3\xhat \,.
\end{align}

Resolving $\vec{BC}$, we get,
\begin{align}
	\vec{BC} & = |\vec{BC}|\cos(60)\xhat - |\vec{BC}|\sin(60)\yhat \nn \\
	         & = \left(4\times \frac{1}{2}\right)\xhat - \left(4\times \frac{\sqrt{3}}{2}\right)\yhat \nn \\
	         & = 2 \xhat - 2\sqrt{3}\yhat \,.
\end{align}

Resolving $\vec{CD}$, we get,
\begin{align}
	\vec{CD} & = -|\vec{CD}|\xhat \nn \\
			 & = -3p\xhat \,.
\end{align}

Resolving $\vec{DA}$, we get,
\begin{align}
	\vec{DA} & = |\vec{DA}|\cos(60)\xhat + |\vec{DA}|\sin(60)\yhat \nn \\
	         & = \left(2q\times \frac{1}{2}\right)\xhat + \left(2q\times\frac{\sqrt{3}}{2}\right)\yhat \nn \\
	         & = q\xhat + q\sqrt{3}\yhat \,.
\end{align}

As the system is in equilibrium, we get that,
\begin{align}
	\sum F \xhat & = 0 \nn \\
	\text{and} \nn \\
	\sum F \yhat & = 0 \,.
\end{align}




\textbf{\textit{Solve and Evaluate:}} \\
Considering the $y$ direction, we get that,
\begin{align}
	\sum F\yhat = 0\yhat & = -2\sqrt{3}\yhat + q\sqrt{3}\yhat \nn \\
	                   0\yhat & = (-2\sqrt{3} +q\sqrt{3})\yhat \nn \\
	        \implies   0 & =  (-2\sqrt{3} +q\sqrt{3}) \nn \\
					q\sqrt{3} & = 2\sqrt{3} \nn \\
				   \implies q & = 2 \,.
\end{align}

Considering the $x$ direction, we get that,
\begin{align}
	\sum F\xhat = 0\xhat & = 3\xhat + 2\xhat -3p\xhat +q\xhat \nn \\
	                   0\xhat & = (3+2-3p+q)\xhat \nn \\
	\implies 0 & = 3+2-3p+q \nn \\
			 3p-q & = 5 \nn \\
	           3p & = 5+q \nn \\
	            p & = \frac{5+q}{3} \nn \\
	              & = \frac{5+2}{3} \nn \\
	              & = \frac{7}{3} \,.
\end{align}


\end{subquestions}