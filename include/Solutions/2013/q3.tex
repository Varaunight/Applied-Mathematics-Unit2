%------------------------------------------------------------------------------
% Author(s):
% \varaun Ramgoolie
% Copyright:
%  Copyright (C) 2020 Brad Bachu, Arjun Mohammed, Varaun Ramgoolie, Nicholas Sammy
%
%  This file is part of Applied-Mathematics-Unit2 and is distributed under the
%  terms of the MIT License. See the LICENSE file for details.
%
%  Description:
%     Year: 2013
%     Module: 2
%     Question: 3 
%------------------------------------------------------------------------------

\begin{subquestions}
	
%------------------------------------------------------------------------------
% 3 a
%------------------------------------------------------------------------------

\subquestion

We can use the properties of $E(X)$ and $\var(X)$ in Section~\ref{mod2:section:ExpectationVariance}.
\begin{subsubquestions}
	
\subsubquestion
We use the property that,
\begin{align}
	E[X+Y] & = E[X] + E[Y] \,.
\end{align}

Substituting the given quantities,
\begin{align}
	E[X+Y] & = 5 + 7 \nn \\
	       & = 12 \,.
\end{align}

%------------------------------------------------------------------------------

\subsubquestion

We use the property that,
\begin{align}
	E[2X-3Y] & = 2 \times E[X] - 3 \times E[Y] \,.
\end{align}

Substituting the given quantities,
\begin{align}
	E[2X-3Y] & = 2 \times 5 - 3 \times 7 \nn \\
	& = 10 - 21 \nn \\
	& = -11 \,.
\end{align}

%------------------------------------------------------------------------------

\subsubquestion

We use the property that,
\begin{align}
	\var[2X-3Y] & = 2^2 \times \var[X] + 3^2 \times \var[Y]
\end{align}

Substituting the given quantities,
\begin{align}
	\var[2X-3Y] & = 4 \times 3 + 9 \times 4 \nn \\
	           & = 12 + 36 \nn \\
	           & = 48 \,.	
\end{align}

\end{subsubquestions}

%------------------------------------------------------------------------------
% 3 b
%------------------------------------------------------------------------------

\subquestion

It is given that,
\begin{equation}
	X \sim \text{Geo}\left(\frac{1}{3}\right).
\end{equation}

\begin{subsubquestions}

\subsubquestion
Since $X$ is discrete, we can write,
\begin{equation}
	P(X\geq 3) = P(X>2) \,.
\end{equation}

Then, using Note ~\ref{mod2:eq:Geometric:Prop}, 
\begin{equation}
	P(X > 2) = (1-p)^2\,.
\end{equation}

Substituting, we find,
\begin{align}
	P(X\geq 3) & = \left(\frac{2}{3} \right)^2 \nn \\
	           & = \frac{4}{9} \,.
\end{align}

%------------------------------------------------------------------------------

\subsubquestion
From \req{mod2:eq:Geometric:Mean},
\begin{equation}
	E(X) = \frac{1}{p} \,.
\end{equation}

Substituting, we find,
\begin{align}
	E(X) & = \frac{1}{\frac{1}{3}} \nn \\
	     & = 3 \,.
\end{align}

%------------------------------------------------------------------------------

\subsubquestion
From \ref{mod2:eq:GeometricDist}, 
\begin{equation}
	P(X=5) = p \times q^{5-1} \,.
\end{equation}

Substituting, we find
\begin{align}
	P(X=5) & = \left(\frac{1}{3} \right) \left(\frac{2}{3} \right)^4 \nn \\
	       & = \left(\frac{1}{3} \right) \left(\frac{16}{81} \right) \nn \\
           & = \frac{16}{243} \,. 
\end{align}

\end{subsubquestions}

%------------------------------------------------------------------------------
% 3 c
%------------------------------------------------------------------------------

\subquestion

We are given that
 $X$ is normally distributed with parameters $\mu = 60$ and $\sigma = 10$.
 \begin{equation}
 	X \sim N(60,10^2)\,.
 \end{equation}
\begin{subsubquestions}
	
\subsubquestion

In order to find $P(X \geq x)$, we must first standardize the random variable $X$ to the normal random variable $Z = \frac{X-\mu}{\sigma}$. Thus, we get that,
\begin{align}
	P(X \geq 65) & = P\left( \frac{X-\mu}{\sigma} \geq \frac{65-\mu}{\sigma} \right) \nn \\
	             & = P\left(Z \geq \frac{65-60}{10} \right) \nn \\
	             & = P\left(Z \geq \frac{5}{10} \right) \nn \\
	             & = P\left(Z \geq \frac{1}{2} \right) \nn \\
	             & = 1 - \Phi\left(\frac{1}{2} \right) \nn \\
	             & = 1 - 0.691 \nn \\
	             & = 0.309 \,.	
\end{align}

%------------------------------------------------------------------------------

\subsubquestion

Similar to \textbf{(3)(c)(i)}, we get that,
\begin{align}
	P(40 \leq X \leq 80) & = P \left( \frac{40-\mu}{\sigma} \leq \frac{X-\mu}{\sigma} \leq \frac{80-\mu}{\sigma} \right) \nn \\
	                     & = P \left( \frac{40-60}{10} \leq Z \leq \frac{80-60}{10} \right) \nn \\
	                     & = P \left( \frac{-20}{10} \leq Z \leq \frac{20}{10} \right) \nn \\
						 & = P(-2 \leq Z \leq 2) \nn \\
						 & = \Phi(2) - \Phi(-2) \nn \\
						 & = \Phi(2) - (1 - \Phi(2)) \nn \\
						 & = 2 \times \Phi(2) - 1 \nn \\
						 & = 2 \times (0.977) - 1 \nn \\
						 & = 0.954 \,.
\end{align}


\end{subsubquestions}

\end{subquestions}

