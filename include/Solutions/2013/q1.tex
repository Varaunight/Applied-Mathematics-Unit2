%------------------------------------------------------------------------------
% Author(s):
% Varaun Ramgoolie
% Copyright:
%  Copyright (C) 2020 Brad Bachu, Arjun Mohammed, Varaun Ramgoolie, Nicholas Sammy
%
%  This file is part of Applied-Mathematics-Unit2 and is distributed under the
%  terms of the MIT License. See the LICENSE file for details.
%
%  Description:
%     Year: 2013
%     Module: 1
%     Question: 1 
%------------------------------------------------------------------------------

\begin{subquestions}

%----------------------------------------------------------------------------
% 1 a 
%----------------------------------------------------------------------------

\subquestion

We can construct the following truth table for the proposition $\boldsymbol{(p \land q) \land \sim (p \lor q)}$.
\begin{table}[ht]
	\centering
	\begin{tabular}{|c|c|c|c|c|c|}
		\hline
		$\boldsymbol{p}$ & $\boldsymbol{q}$ & $\boldsymbol{p \land q}$ & $\boldsymbol{p \lor q}$ & $\boldsymbol{\sim (p \lor q)}$ & $\boldsymbol{(p \land q) \land \sim (p \lor q)}$\\
		\hline
		0 & 0 & 0 & 0 & 1 & 0 \\
		0 & 1 & 0 & 1 & 0 & 0 \\
		1 & 0 & 0 & 1 & 0 & 0 \\
		1 & 1 & 1 & 1 & 0 & 0 \\
		\hline
	\end{tabular}
	\caption{\label{2013:q1:tab:TruthTab1} Truth Table of $\boldsymbol{(p \land q) \land \sim (p \lor q)}$\,.}
\end{table}

%------------------------------------------------------------------------------
% 1 b
%------------------------------------------------------------------------------

\subquestion

From \rdef{mod1:defn:Contradiction}, we see that $\boldsymbol{(p \land q)\, \land \sim (p \lor q)}$ is a contradiction since the truth value for the statement is always 0.

%------------------------------------------------------------------------------
% 1 c
%------------------------------------------------------------------------------

\subquestion

Using De Morgan's Law (\rdef{mod1:law:DeMorgan}),we can write a logically equivalent proposition as follows:

\begin{equation}
	\boldsymbol{\sim (p \lor q) \equiv \, \sim p \, \land \sim q} \,.
\end{equation}

%------------------------------------------------------------------------------
% 1 d
%------------------------------------------------------------------------------

\subquestion

Let $p$="It is sunny" and let $q$="It is hot". 
 
Therefore, the statement "It is not true that it is hot and sunny" can be expressed as the Boolean proposition $\boldsymbol{\sim (p \land q)} \,.$ \\

Using De Morgan's Law (\rdef{mod1:law:DeMorgan}) again,
\begin{equation}
	\boldsymbol{\sim (p \land q) \equiv \, \sim p \, \lor \sim q} \,.
\end{equation}

Thus, the equivalent statement of 	$\boldsymbol{(\sim p \, \lor \sim q)}$ is "It is not hot or it is not sunny."

%------------------------------------------------------------------------------
% 1 e
%------------------------------------------------------------------------------

\subquestion

\begin{subsubquestions}
	 
\subsubquestion

We can draw the switching circuit given by the Boolean expression 
\begin{equation}
	\boldsymbol{p \land q \land (p \lor r) \land (q \lor (r \land p) \lor s}
\end{equation} as follows.

\begin{center}

\begin{circuitikz}[scale=0.75]
		
	\draw (0,0) to[normal open switch, *-*](3,0);
	\path (0,0) -- (3,0) node[pos=0.5,below]{p};
	
	\draw (3,0) to[normal open switch, *-*](6,0);
	\path (3,0) -- (6,0) node[pos=0.5,below]{q};
	
	\draw [color=black, thin] (6,0) -- (6,1);
	\draw [color=black, thin] (6,0) -- (6,-1);
	
	\draw (6,1) to[normal open switch, *-*](9,1);
	\path (6,1) -- (9,1) node[pos=0.5, below]{p};
	
	\draw (6,-1) to[normal open switch, *-*](9,-1);
	\path (6,-1) -- (9,-1) node[pos=0.5, below]{r};
	
	\draw [color=black, thin] (6,1) -- (6,0);
    \draw [color=black, thin] (6,-1) -- (6,0);	
    
  	\draw [color=black, thin] (9,0) -- (9,1);
    \draw [color=black, thin] (9,0) -- (9,-1);

	\draw [color=black, thin] (9,0) -- (11,0);
	\draw [color=black, thin] (11,0) -- (11,1);
	\draw [color=black, thin] (11,0) -- (11,-1);
		
	\draw (11,0) to[normal open switch, *-*](14,0);
	\draw (11,1) to[normal open switch, *-*](14,1);
	\draw (11,-1) to[normal open switch, *-*](12.5,-1);
	\draw (12.5,-1) to[normal open switch, *-*](14,-1);
	
	\path (11,0) -- (14,0) node[pos=0.5,below]{s};
	\path (11,1) -- (14,1) node[pos=0.5,below]{q};
	\path (11,-1) -- (12.5,-1) node[pos=0.5,below]{r};
	\path (12.5,-1) -- (14,-1) node[pos=0.5,below]{p};
	
	\draw [color=black, thin] (14,1) -- (14,0);
	\draw [color=black, thin] (14,-1) -- (14,0);
	
	\draw [color=black, thin] (14,0) -- (15,0);	
\end{circuitikz}
	
\end{center}

%-----------------------------------------------------------------------------

\subsubquestion

Using the Commutative Law (\rdef{mod1:law:Commutative}), we can manipulate the expression as follows,
\begin{align}
	\boldsymbol{p \land q \land (p \lor r) \land (q \lor (r \land p) \lor s)
	  \equiv p \land (p \lor r) \land q \land (q \lor (r \land p) \lor s)} \,. \label{2013:q1:BooleanEquation}
\end{align}

Let $\boldsymbol{A \equiv p \land (p \lor r)}$ and $\boldsymbol{B \equiv q \land (q \lor (r \land p) \lor s)}$. \\

Using the Absorptive Law (\rdef{mod1:law:Absorptive}), we can simplify $\boldsymbol{A}$ as,
\begin{align}
	\boldsymbol{A} & \boldsymbol{\equiv p \land (p \lor r)} \nn \\
	  & \boldsymbol{\equiv p} \,.
\end{align}

Similarly, using the Absorptive Law (\rdef{mod1:law:Absorptive}) twice, $B$ can be simplified as,
\begin{align}
	\boldsymbol{B} & \boldsymbol{\equiv q \land (q \lor (r \land p) \lor s)} \nn \\
	  & \boldsymbol{\equiv q \land (q \lor s)} \nn \\
	  & \boldsymbol{\equiv q} \,.
\end{align}

Thus, \req{2013:q1:BooleanEquation} can be simplified to,
\begin{align}
	\boldsymbol{p \land (p \lor r) \land q \land (q \lor (r \land p) \lor s) \equiv p \land q} \,.
\end{align}

\end{subsubquestions}

%------------------------------------------------------------------------------
% 1 f
%------------------------------------------------------------------------------

\subquestion

We can draw the following logic gates circuit to represent $\boldsymbol{\sim((p \land q) \lor r)}$.
\begin{figure}[H]
	\begin{center}
		\includegraphics{../2013/figures/2013q1LogicGate}
		\caption{\label{2013:q1:fig:LogicGates} Logic Gate representation of $\boldsymbol{\sim((p \land q) \lor r)}$\,.}
	\end{center}
\end{figure}

\end{subquestions}

