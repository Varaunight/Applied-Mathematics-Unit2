%------------------------------------------------------------------------------
% Author(s):
% Varaun Ramgoolie
% Copyright:
%  Copyright (C) 2020 Brad Bachu, Arjun Mohammed, Varaun Ramgoolie, Nicholas Sammy
%
%  This file is part of Applied-Mathematics-Unit2 and is distributed under the
%  terms of the MIT License. See the LICENSE file for details.
%
%  Description:
%     Year: 2013
%     Module: 2
%     Question: 4 
%------------------------------------------------------------------------------

\begin{subquestions}
	
%------------------------------------------------------------------------------
% 4 a
%------------------------------------------------------------------------------

\subquestion

It is given that accidents occur on a certain highway at an average rate of 1 per week. This can be modeled using a Poisson distribution.

Let $X$ be the discrete random variable representing the number of accidents which occur on that highway per week. We can express this as $X \sim $ Pois$(1)$.

\begin{subsubquestions}
	
\subsubquestion

We want to find $P(X=2)$. Using \rdef{mod2:defn:Poisson}, we know that,
\begin{equation}
	P(X=k)= \frac{1^k \times e^{-1}}{k!} \,.
\end{equation}
	
Therefore, we can find $P(X=2)$ as follows,
\begin{align}
	P(X=2) & = \frac{1^2 \times e^{-1}}{2!} \nn \\
	       & = \frac{1 \times e^{-1}}{2} \nn \\
	       & = \frac{1}{2 \times e} \nn \\
	       & = 0.184 \,.
\end{align}
	
%------------------------------------------------------------------------------

\subsubquestion

We are now given that an average of 4 accidents occur in a 4-week period. 

Let $Y$ be the discrete random variable representing the number of accidents which occur on that highway during a 4-week period. We can express this as $Y \sim $ Pois$(4)$.

We want to show that $P(Y \geq 4) = 0.567$. Using \req{mod2:eq:PoissonDist}, we know that,
\begin{equation}
	P(Y=k)= \frac{4^k \times e^{-4}}{k!} \,. \label{2013:q4:eq:Pois1}
\end{equation}

We can notice that,
\begin{align}
	P(Y \geq 4) & = 1 - P(Y \leq 3) \nn \\
	            & = 1 - \left[P(Y=0)+P(Y=1)+P(Y=2)+P(Y=3) \right] \,.
\end{align}

Using \req{2013:q4:eq:Pois1}, we can calculate $P(Y \leq 3)$ as follows,
\begin{align}
	P(Y=0) & = \frac{4^0 \times e^{-4}}{0!} 
	         = \frac{1 \times e^{-4}}{1} \nn \\
	       & = \frac{1}{e^4} \,. \\	\nn \\
	P(Y=1) & = \frac{4^1 \times e^{-4}}{1!} 
	         = \frac{4 \times e^{-4}}{1} \nn \\
	       & = \frac{4}{e^4} \,. \\	\nn \\       
	P(Y=2) & = \frac{4^2 \times e^{-4}}{2!} 
		     = \frac{16 \times e^{-4}}{2} \nn \\
           & = \frac{8}{e^4} \,. \\ \nn \\            
	P(Y=3) & = \frac{4^3 \times e^{-4}}{3!} 
             = \frac{64 \times e^{-4}}{6} \nn \\
           & = \frac{64}{6 \times e^4} \,. \\ \nn \\         
    P(Y \leq 3) & = P(Y=0)+P(Y=1)+P(Y=2)+P(Y=3) \nn \\
                & = \frac{1}{e^4} + \frac{4}{e^4} + \frac{8}{e^4} + \frac{64}{6 \times e^4} \nn \\
                & = \frac{1}{e^4} \times \left(1+4+8+\frac{64}{6} \right) \nn \\
                & = \frac{1}{e^4} \times \left(\frac{142}{6} \right) \nn \\
                & = \frac{142}{6 \times e^4} \nn \\
                & = 0.433 \,.
\end{align}

Thus,
\begin{align}
	P(Y \geq 4) & = 1 - P(Y \leq 3) \nn \\
	            & = 1 - 0.433 \nn \\
	            & = 0.567 \,.
\end{align}

%------------------------------------------------------------------------------

\subsubquestion

We want to find the probability that, in a year consisting of 13 4-week periods, exactly 11 4-week periods have at least 4 accidents. This can be found using a Binomial Distribution. Let $Z$ be the number of 4-week periods in which at least 4 accidents occurred. From \textbf{(4)(a)(ii)}, we know that the probability that at least 4 accidents occur in a 4-week period is 0.567.

From this information and \rdef{mod2:defn:Binomial}, the probability mass function, $P(Z=z)$, can be expressed as,
\begin{equation}
		P(Z = z) = { 13 \choose z} (0.567)^z  (1-0.567)^{13-z} \,.
\end{equation}

Thus, to find the probability that exactly 11 4-week periods have at least 4 accidents, we set $z=11$ and calculate it as follows,
\begin{align}
	P(Z = 11) & = { 13 \choose 11} (0.567)^{11}  (1-0.567)^{13-11} \nn \\
	          & = { 13 \choose 11} (0.567)^{11}  (0.433)^{2} \nn \\
	          & = 78 \times (0.567)^{11} \times (0.433)^{2} \nn \\
	          & = 0.028 \,.
\end{align}

\end{subsubquestions}

%------------------------------------------------------------------------------
% 4 b
%------------------------------------------------------------------------------

\subquestion

We are given that nails are packaged in boxes of 100 and the probability that a nail is defective is 0.02. We want to find the probability that there are at most 2 defective nails in a box (using a justified probability distribution approximation). Let $X$ be the number of defective nails in a box.

From the information given, we know that $n=100$ and $p=0.02$. From ~\ref{mod2:section:PoissonApproxBinomial}, we can notice that $n=100 \geq 15$ and $np= 100 \times 0.02=2<15.$ Thus, we can approximate the probability using a Poisson Distribution as follows,
\begin{align}
	& X \sim \text{Pois}(np) \nn \\
	\Longrightarrow & X \sim \text{Pois}(2) \,.
\end{align}

Using \rdef{mod2:defn:Poisson}, the probability mass function of $X$ is,
\begin{equation}
	P(X=x)= \frac{2^x \times e^{-2}}{x!} \,.
\end{equation}

We know that,
\begin{equation}
	P(X \leq 2) = P(X=0)+P(X=1)+P(X=2) \,.
\end{equation}

We can now begin calculating $P(X \leq 2)$ as follows,
\begin{align}
	P(Y=0) & = \frac{2^0 \times e^{-2}}{0!} 
             = \frac{1 \times e^{-2}}{1} \nn \\
	       & = \frac{1}{e^2} \,. \\	\nn \\
	P(Y=1) & = \frac{2^1 \times e^{-2}}{1!} 
	         = \frac{2 \times e^{-2}}{1} \nn \\
	       & = \frac{2}{e^2} \,. \\	\nn \\       
	P(Y=2) & = \frac{2^2 \times e^{-2}}{2!} 
	         = \frac{4 \times e^{-2}}{2} \nn \\
	       & = \frac{2}{e^2} \,. \\ \nn \\     
	P(X \leq 2) & = P(X=0)+P(X=1)+P(X=2) \nn \\
	            & = \frac{1}{e^2} + \frac{2}{e^2} + \frac{2}{e^2} \nn \\
	            & = \frac{5}{e^2} \nn \\
	            & = 0.677 \,. 
\end{align}

%------------------------------------------------------------------------------
% 4 c
%------------------------------------------------------------------------------

\subquestion

It is given that the continuous random variable, $X$, has a uniform distribution given as,
\[
f(x) =
\begin{cases}
	k & \text{$0 \leq x \leq 3$} \\
	0    & \text{otherwise} \\
\end{cases}
\]

\begin{subsubquestions}
	
\subsubquestion

From \rprop{mod2:prop:ContinuousRV:1}, we know that,
\begin{equation}
	\int_{-\infty}^{\infty} f(x).dx = 1
\end{equation}

Thus, we get that,
\begin{align}
	\int_{0}^{3} k.dx & = 1 \,, \nn \\
	\left[\frac{kx}{1} \right]_0^3 & = 1 \,, \nn \\
	[k(3)]-[k(0)] & = 1 \,, \nn \\
	3k & = 1 \,, \nn \\
	\implies k & = \frac{1}{3} \,.
\end{align}

Thus,
\[
f(x) =
\begin{cases}
	\frac{1}{3} & \text{$0 \leq x \leq 3$} \\
	0    & \text{otherwise} \\
\end{cases}
\]
%------------------------------------------------------------------------------

\subsubquestion

We are given that $P(X>t) = \frac{1}{4}$. Since $f(x)$ is continuous between 0 and 3, we can then notice that $P(X<t)=1-\frac{1}{4}=\frac{3}{4}$.

From \rdef{mod2:defn:ContinuousRV:CDF}, we know that the cumulative distribution function, $F(x)$ is defined as $P(X<x)$. Thus, to solve for $t$, we must first find $F(x)$.

Using \rdef{mod2:defn:ContinuousRV:CDF},
\begin{align}
	F(x) & = \int_{-\infty}^{x} f(x).dx \nn \\
	     & = \int_{0}^{x} \frac{1}{3}.dx \nn \\
	     & = \left[\frac{x}{3} \right]_0^x \nn \\
	     & = \left[\frac{x}{3} \right] - \left[\frac{0}{3} \right] \nn \\
	     & = \frac{x}{3} \,.
\end{align}

Hence, $F(x)=P(X<x)=\frac{x}{3}$. Therefore, we can now input $x=t$ and solve for $t$ as follows,
\begin{align}
	P(X<t)=  F(t) & =\frac{3}{4}\,, \nn \\
	        \frac{t}{3} & = \frac{3}{4} \,. \nn \\
	        \implies  t & = \frac{3 \times 3}{4} \nn \\
	                    & = \frac{9}{4} \,.
\end{align}

\end{subsubquestions}

\end{subquestions}

