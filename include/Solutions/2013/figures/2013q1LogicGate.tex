%------------------------------------------------------------------------------
% Author(s):
% Varaun Ramgoolie
% Copyright:
%  Copyright (C) 2020 Brad Bachu, Arjun Mohammed, Varaun Ramgoolie, Nicholas Sammy
%
%  This file is part of Applied-Mathematics-Unit2 and is distributed under the
%  terms of the MIT License. See the LICENSE file for details.
%
%  Description:
%     Circuit for 2013 q1 f
%     
%     
%------------------------------------------------------------------------------

\documentclass[crop,tikz]{standalone}

\usepackage{../../../../src/tikzappmath}

\usetikzlibrary{scopes}
\usetikzlibrary{arrows,shapes.gates.logic.US,shapes.gates.logic.IEC,calc}


\begin{document}
	
	\begin{tikzpicture}
		
		\def\xspace{3} % set the horizonal distance between gates
		\def\yspace{2} % set the vertical distance between gates
		\def\shift{0.5cm} % sets some spacing to draw connecting lines
		
		% first set of gates
		\node[and gate US,gatestyle] at (0,0) (x1) {};

		% second set of gates
		\node[or gate US, gatestyle] at (2,-1) (x2) {};
		\node[not gate US, gatestyle] at (5,-1) (x3){};
		
		% connections between gates
		\draw (x1.output) -| ([xshift = -\shift]x2.input 1) --++(\shift,0);
		\draw (x2.output) -| ([xshift = -\shift]x3.input) --++(\shift,0);
		
		% inputs for first gate
	    \draw (x1.input 1) -|++ (-\shift,\shift) --++(-\shift,0) node[left,]{${p}$};
		\draw (x1.input 2) -|++ (-\shift,-\shift) --++(-\shift,0) node[left]{$q$};
		
		% inputs for second gate
		\draw (x2.input 2) -|++ (-5.2*\shift,0) --++(-\shift,0) node[left]{$r$};
		
		% end piece at last gate
		\draw (x3.output) --++ (3*\shift,0);
		
		% naming the expressions at the top of the gates
		\node[above] at ($(x1.output) + (\shift,0)$) (l1) {\gatetext${p \land q}$};
	    \node[above] at ($(x2.output) + (\shift,0)$) (l1) {\gatetext${\,\,\,(p \land q) \lor r}$};	
		\node[above] at ($(x3.output) + (\shift,0)$) (l1) {\gatetext$\sim((p \land q) \lor r)$};
	
	\end{tikzpicture}
	
\end{document}