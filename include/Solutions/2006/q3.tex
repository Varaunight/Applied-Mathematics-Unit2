%------------------------------------------------------------------------------
% Author(s):
% Varaun Ramgoolie
% Copyright:
%  Copyright (C) 2020 Brad Bachu, Arjun Mohammed, Varaun Ramgoolie, Nicholas Sammy
%
%  This file is part of Applied-Mathematics-Unit2 and is distributed under the
%  terms of the MIT License. See the LICENSE file for details.
%
%  Description:
%     Year: 2006 C
%     Module: 2
%     Question: 3
%------------------------------------------------------------------------------

%------------------------------------------------------------------------------
% 3 a
%------------------------------------------------------------------------------

\begin{subquestions}
	
\subquestion

We are given the cumulative distribution function, $F(x)$, of a continuous random variable $X$.
	
\begin{subsubquestions}
	
\subsubquestion

From \rprop{mod2:prop:ContinuousRV:2}, we know that $P(X=a)$ for any real $a$ is $0$. So,
\begin{equation}
	P\left(X= \frac{1}{2}\right) = 0 \,.
\end{equation}
	
%------------------------------------------------------------------------------

\subsubquestion

We can compute probabilities via the c.d.f using Note ~\ref{mod2:note:ContinuousRV:CDF}, 
\begin{align}
	P\left(\frac{1}{2} < X \leq \frac{3}{4}\right) & = F\left(\frac{3}{4}\right) - F\left(\frac{1}{2}\right) \,.
\end{align}

Substituting the given c.d.f yields,
\begin{align}
P\left(\frac{1}{2} < X \leq \frac{3}{4}\right)	                                    & = \frac{\left(\frac{3}{4}\right)^2+\frac{3}{4}}{2} -   \frac{\left(\frac{1}{2}\right)^2+\frac{1}{2}}{2} \nn \\
	                                    & = \frac{\frac{9}{16}+\frac{3}{4}}{2} - \frac{\frac{1}{4}+\frac{1}{2}}{2} \nn \\
	                                    & = \frac{\frac{21}{16}}{2} - \frac{\frac{3}{4}}{2} \nn \\
	                                    & = \frac{\frac{21}{16} - \frac{3}{4}}{2} \nn \\
	                                    & = \frac{\frac{9}{16}}{2} \nn \\
	                                    & = \frac{9}{32} \,.
\end{align}

%------------------------------------------------------------------------------

\subsubquestion

We know that the c.d.f, $F$, and p.d.f, $f$, are related in \rdef{mod2:defn:ContinuousRV:CDF}, via
\begin{align}
	f(x) & = \ddd{}{x} F(x)
\end{align}

Now, we must compute the p.d.f in the three regions over which $F$ is defined. The only non-trivial calculation is when $0 \leq x < 1$,
\begin{align}
f(x)  & = \ddd{}{x}\left(\frac{x^2+x}{2} \right) \nn \\
	   & = x + \frac{1}{2} \,.
\end{align}

Thus, $f(x)$ is defined as,
\begin{equation}
f(x) = 
\begin{cases}
	0, & x<0 \\
	x + \frac{1}{2}, & 0 \leq  x < 1 \\
	0, & x \geq 1 \\
\end{cases}
\end{equation}

\end{subsubquestions}	
	
%------------------------------------------------------------------------------
% 3 a
%------------------------------------------------------------------------------

\subquestion

The probability density function, $f(y)$, of the continuous random variable, $Y$, is given.

\begin{subsubquestions}
	
\subsubquestion

We can use the property that probabilities must sum to one, \rprop{mod2:prop:ContinuousRV:1} for continuous random variables,
\begin{align}
	\int_{-\infty}^{\infty} f(y)\, \dd y & = 1 \,.
\end{align}

Performing this integral requires us to split the integral into three regions over which $f$ is defined,
\begin{align}
\int_{-\infty}^{\infty} f(y)\, \dd y & = \int_{-\infty}^{1} f(y)\, \dd y + \int_{1}^{2} f(y)\, \dd y + \int_{2}^{\infty} f(y)\, \dd y
\end{align}

Since we know how the functions are defined in each region, we can solve for $k$,
\begin{align}
	0 + \int_{1}^{2} \left(ky^3\right)\, \dd y + 0 & = 1 \nn \\
	k \times \int_{1}^{2} \left(y^3\right)\mathrm{d}y & = 1 \nn \\
	k \times \left[\frac{y^4}{4} \right]^2_1 & = 1 \nn \\
	k \times \left[\frac{2^4}{4} - \frac{1^4}{4}\right] & = 1 \nn \\
	k \times \left[\frac{16}{4}- \frac{1}{4}\right] & = 1 \nn \\
	\frac{15k}{4} & = 1 \nn \\
	\implies k & = \frac{4}{15} \,.
\end{align}
	
%------------------------------------------------------------------------------

\subquestion

From \rdef{mod2:defn:ContinuousRV:Expectation}, we know that the expectation of a continuous random variable $Y$ is given by,
\begin{align}
	E(Y) & = \int_{-\infty}^{\infty}y f(y)\mathrm{d}y 
\end{align}

Splitting up the integral as before,
\begin{align}
E(Y)	& = \int_{-\infty}^{1}y f(y)\, \dd y + \int_{1}^{2}y f(y)\, \dd y + \int_{2}^{\infty}y f(y)\, \dd y \,,
\end{align}
allows us to evaluate
\begin{align}
E(Y) 	& = 0 + \int_{1}^{2} \left( \frac{4y^4}{15}\right)\mathrm{d}y + 0 \nn \\
		& = \frac{4}{15} \times \left[\frac{y^5}{5} \right]^2_1 \nn \\
	   & = \frac{4}{15} \times \left[\frac{2^5}{5} -\frac{1^5}{5} \right] \nn \\
	   & = \frac{4}{15} \times \left[\frac{32}{5} -\frac{1}{5} \right] \nn \\
	   & = \frac{124}{75} \,. 
\end{align}

%------------------------------------------------------------------------------

\subquestion
We want to find the value of $y_{30}$ such that,
\begin{align}
	P(Y \leq y_{30}) &= \frac{30}{100}
\end{align}

From Note ~\ref{mod2:note:ContinuousRV:Note1}, we know that given the p.d.f., we can find the probability by integrating $f(y)$ in the appropriate region,
\begin{align}
	P(Y \leq y_{30}) &= \int_{-\infty}^{y_{30}} f(y) \,\dd y \,.
\end{align}

Since $f(y)$ is defined piecewise, we must split the integral up accordingly,
\begin{align}
P(Y \leq y_{30}) &= \int_{-\infty}^1 f(y) \, \dd y + \int_{1}^{y_{30}} f(y) \,\dd y
\end{align}

Evaluating the integral over the different regions yields,
\begin{align}
	P(Y \leq y_{30}) & = 0 +  \int_{1}^{y_{30}} f(y)\mathrm{d}y  \nn \\
	                   &=\frac{4}{15} \times \left[\frac{y^4}{4}\right]^{{y_{30}}}_1 \nn \\
	                   &= \frac{4}{15} \times \left[\frac{{y_{30}}^4}{4} - \frac{1^4}{4}\right]  \nn \\
	                   &= \frac{1}{15} \times \left[{y_{30}}^4 - 1\right]   
\end{align}

Now, we can solve for $y_{30}$ by recalling that $P(Y \leq y_{30}) = 0.3$,
\begin{align}                 
	\frac{1}{15} \times \left[{y_{30}}^4 - 1\right] &= 0.03 \nn \\
	                                  {y_{30}}^4 -1 & = 4.5 \nn \\
	                   \implies {y_{30}} & = \sqrt[4]{5.5} \nn \\
	                                     & \approx 1.531 \,.             
\end{align}

\end{subsubquestions}
	
\end{subquestions}


