%------------------------------------------------------------------------------
% Author(s):
% Varaun Ramgoolie
%
% Copyright:
%  Copyright (C) 2020 Brad Bachu, Arjun Mohammed, Varaun Ramgoolie, Nicholas Sammy
%
%  This file is part of Applied-Mathematics-Unit2 and is distributed under the
%  terms of the MIT License. See the LICENSE file for details.
%
%  Description:
%     Year: 2006 C
%     Module: 3
%     Question: 6
%------------------------------------------------------------------------------

%------------------------------------------------------------------------------
% 6 a
%------------------------------------------------------------------------------

\begin{subquestions}
	
\subquestion

\textbf{\textit{Diagram:}} \\
\addimage{../2006/figures/2006q6-Diagram1}{2006:q6:Diagram1}{Forces acting on the bicycle}

%------------------------------------------------------------------------------

\subquestion

\textbf{\textit{Sketch and Translate:}} \\
As we are given that the resistive force, $F_R$, is directly proportional to the square of the speed, $v^2$, we get that $|F_R|=kv^2$. \\




\textbf{\textit{Simplify and Diagram:}} \\
\addimage{../2006/figures/2006q6-Diagram2}{2006:q6:Diagram2}{Forces acting on the bicycle}. We will first define the following forces:
\begin{itemize}
	\item $\vec{F_T}$ is the forward thrust of the bike, $|F_T|=180N$,
	\item $\vec{F_R}$ is the resistive force, $|F_R|=kv^2$,
	\item $\vec{F}$ is the resultant force on the bike, $\vec{F}=m\vec{a}$.
\end{itemize}
We will use Newton's Second Law to find an expression for $\vec{F}$ and work towards our goal.\\




\textbf{\textit{Represent Mathematically:}} \\
By Newton's Second Law, know that,
\begin{align}
	\sum F = ma & = F_T - F_R \nn \\
           & = 180-kv^2 \,.	\label{2006:q6:FEqn1} 
\end{align}
We should note that $F_R$ is attached to a negative sign as it is a resistive force (its direction is opposite to the direction of motion of the body).

We can express our acceleration as,
\begin{equation}
	a = v\ddd{v}{s} \label{2006:q6:AEqn1} \,.
\end{equation}

We are also given that,
\begin{align}
	m=70kg \,, \nn \\
	v(t=0) & = 0 \,, \nn \\
	a(v=6) & = 0 \label{2006:q6:AEqn2} \,.
\end{align}





\textbf{\textit{Solve and Evaluate:}} \\
Considering \reqs{2006:q6:FEqn1}{2006:q6:AEqn1}, we get that,
\begin{align}
	70v\ddd{v}{s} & = 180-kv^2 \nn \\
	\implies \ddd{s}{v} & = \frac{70v}{180-kv^2} \label{2006:q6:DEqn1} \,.
\end{align}

Considering \reqs{2006:q6:FEqn1}{2006:q6:AEqn2}, we get that,
\begin{align}
	70(0) & = 180 - k(6)^2 \nn \\
	\implies k & = 5  \,.
\end{align}

Substituting this value of $k$ into \req{2006:q6:DEqn1}, we get,
\begin{align}
	\ddd{s}{v} & = \frac{70v}{180-5v^2} \nn \\
	           & = \frac{5(14v)}{5(36-v^2)} \nn \\
	           & = \frac{14v}{36-v^2} \,.
\end{align} 

\end{subquestions}