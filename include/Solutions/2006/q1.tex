%------------------------------------------------------------------------------
% Author(s):
% Varaun Ramgoolie
% Copyright:
%  Copyright (C) 2020 Brad Bachu, Arjun Mohammed, Varaun Ramgoolie, Nicholas Sammy
%
%  This file is part of Applied-Mathematics-Unit2 and is distributed under the
%  terms of the MIT License. See the LICENSE file for details.
%
%  Description:
%     Year: 2006 C
%     Module: 1
%     Question: 1 
%------------------------------------------------------------------------------

\begin{subquestions}

%------------------------------------------------------------------------------
% 1 a--------------------------------------------------------------------------
%------------------------------------------------------------------------------
 
\subquestion
Given the proposition $\boldsymbol{x \implies y}$,

\begin{subsubquestions}

%------------------------------------------------------------------------------

\subsubquestion

$\boldsymbol{(\sim x \implies \sim y)}$ is the Inverse.

%------------------------------------------------------------------------------

\subsubquestion

$\boldsymbol{(y \implies x)}$ is the Converse.

%------------------------------------------------------------------------------

\subsubquestion

$\boldsymbol{(\sim y \implies \sim x)}$ is the Contrapositive.

\end{subsubquestions}
	
%------------------------------------------------------------------------------
% 1 b--------------------------------------------------------------------------
%------------------------------------------------------------------------------

\subquestion

\begin{subsubquestions}

%-----------------------------------------------------------------------------

\subsubquestion

We can complete the truth table as follows

\begin{table}[ht]
	\centering
	\begin{tabular}{|c|c|c|c|c|c|}
		\hline
		$\boldsymbol{x}$ & $\boldsymbol{y}$ & $\boldsymbol{\sim x}$ & $\boldsymbol{\sim y}$ & $\boldsymbol{x \implies y}$ & $\boldsymbol{\sim y \implies \sim x}$ \\
		\hline
		0 & 0 & 1 & 1 & 1 & 1 \\
		0 & 1 & 1 & 0 & 1 & 1 \\
		1 & 0 & 0 & 1 & 0 & 0 \\
		1 & 1 & 0 & 0 & 1 & 1 \\
		\hline
	\end{tabular}
	\caption{\label{2006:q1:tab:TruthTab1} Truth Table of $\boldsymbol{x \implies y}$ and $\boldsymbol{\sim y \implies \sim x}$.}
\end{table}

%-----------------------------------------------------------------------------
	
\subsubquestion

From \rtab{2006:q1:tab:TruthTab1}, we can see that the truth values of $\boldsymbol{x \implies y}$ and $\boldsymbol{\sim y \implies \sim x}$ are the same. Therefore, both expressions are logically equivalent.

\end{subsubquestions}

%------------------------------------------------------------------------------
% 1 c--------------------------------------------------------------------------
%------------------------------------------------------------------------------

\subquestion

\begin{subsubquestions}

%------------------------------------------------------------------------------

\subsubquestion

If $\boldsymbol{p}$ is the proposition "You have your cake" which is equivalent to "You do not eat your cake", then "You have your cake and eat it" can be expressed as,
\begin{equation}
	\boldsymbol{p ~\land \sim p} \,.
\end{equation}

%------------------------------------------------------------------------------

\subsubquestion

The following truth can be constructed for the proposition $\boldsymbol{p \land \sim p}$,

\begin{table}[ht]
	\centering
	\begin{tabular}{|c|c|c|}
		\hline
		$\boldsymbol{p}$ & $\boldsymbol{\sim p}$ & $\boldsymbol{p ~\land \sim p}$ \\
		\hline
		0 & 1 & 0 \\
		1 & 0 & 0 \\ 
		\hline
	\end{tabular}
	\caption{\label{2006:q1:tab:TruthTab2} Truth Table of $\boldsymbol{p ~\land \sim p}$.}
\end{table}

From \rtab{2006:q1:tab:TruthTab2} and \rdef{mod1:defn:Contradiction}, we can see that $\boldsymbol{p ~\land \sim p}$ is a contradiction because the truth table is alwaays 0.

\end{subsubquestions}

%------------------------------------------------------------------------------
% 1 d--------------------------------------------------------------------------
%------------------------------------------------------------------------------

\subquestion

We can construct the following activity network from the information given,
\begin{figure}[H]
	\begin{center}
		\includegraphics{../2006/figures/2006q1ActNet}
		\caption{\label{2006:q1:fig:ActNet} Activity Network of the project.}
	\end{center}
\end{figure}

\end{subquestions}

