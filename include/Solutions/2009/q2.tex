%------------------------------------------------------------------------------
% Author(s):
% Vraaun Ramgoolie
% Copyright:
%  Copyright (C) 2020 Brad Bachu, Arjun Mohammed, Varaun Ramgoolie, Nicholas Sammy
%
%  This file is part of Applied-Mathematics-Unit2 and is distributed under the
%  terms of the MIT License. See the LICENSE file for details.
%
%  Description:
%     Year: 2009
%     Module: 1
%     Question: 2 
%------------------------------------------------------------------------------

\begin{subquestions}

%------------------------------------------------------------------------------
% 2 a--------------------------------------------------------------------------
%------------------------------------------------------------------------------

\subquestion

If $\boldsymbol{p}$ is the propositon "Tom works hard" and $\boldsymbol{q}$ is the proposition "Tom is successful", we can express $\boldsymbol{p \iff q}$ in words as "Tom works hard if and only if Tom is successful."

%------------------------------------------------------------------------------
% 2 b--------------------------------------------------------------------------
%------------------------------------------------------------------------------
	
\subquestion 

Given the statement $\boldsymbol{p \implies q}$, we can write the:
\begin{subsubquestions}
	
%-----------------------------------------------------------------------------

\subsubquestion
Inverse
\begin{equation}
	\boldsymbol{\sim p \implies \sim q}\,.
\end{equation}

%-----------------------------------------------------------------------------

\subsubquestion
Converse
\begin{equation}
	\boldsymbol{q \implies p}\,.
\end{equation}

%-----------------------------------------------------------------------------

\subsubquestion
Contrapositive
\begin{equation}
	\boldsymbol{\sim q \implies \sim p}\,.
\end{equation}

\end{subsubquestions}

%------------------------------------------------------------------------------
% 2 c--------------------------------------------------------------------------
%------------------------------------------------------------------------------

\subquestion

We can construct the following truth table for the proposition $\boldsymbol{(a \land (a \implies b)) \land \sim b}$.
\begin{table}[ht]
	\centering
	\begin{tabular}{|c|c|c|c|c|c|}
		\hline
		$\boldsymbol{a}$ & $\boldsymbol{b}$ & $\boldsymbol{\sim b}$ & $\boldsymbol{a \implies b}$ & $\boldsymbol{a \land (a \implies b)}$ & $\boldsymbol{(a \land (a \implies b)) \land \sim b}$ \\
		\hline
		0 & 0 & 1 & 1 & 0 & 0 \\
		0 & 1 & 0 & 1 & 0 & 0 \\
		1 & 0 & 1 & 0 & 0 & 0 \\
		1 & 1 & 0 & 1 & 1 & 0 \\
		\hline
	\end{tabular}
	\caption{\label{2009:q2:tab:TruthTab1} Truth table of $\boldsymbol{(a \land (a \implies b)) ~\land \sim b}$.}
\end{table}

From the truth table, it can be seen that the proposition $\boldsymbol{(a \land (a \implies b)) ~\land \sim b}$ is a contradiction because its truth value is always 0.

%------------------------------------------------------------------------------
% 2 d--------------------------------------------------------------------------
%------------------------------------------------------------------------------

\subquestion

From $A$ to $E$, the possible paths are,
\begin{align}
	& A \rightarrow E\,, \nn \\
	& A \rightarrow D \rightarrow E\,, \nn \\
	& A \rightarrow B \rightarrow D \rightarrow E\,, \nn \\
	& A \rightarrow B \rightarrow C \rightarrow E\,, \nn \\
	& A \rightarrow B \rightarrow C \rightarrow D \rightarrow E\,, \nn \\
	& A \rightarrow B \rightarrow D \rightarrow C \rightarrow E\,. 	
\end{align}

%------------------------------------------------------------------------------
% 2 e--------------------------------------------------------------------------
%------------------------------------------------------------------------------

\subquestion

The degree of vertex $D$ is $4$.

%------------------------------------------------------------------------------
% 2 f--------------------------------------------------------------------------
%------------------------------------------------------------------------------

\subquestion

The Boolean expression $\boldsymbol{(x \land y) ~\lor \sim y}$ can be expressed as the following logic gate circuit.
\begin{figure}[H]
	\begin{center}
		\includegraphics{../2009/figures/2009q2LogicGate}
		\caption{\label{2009:q2:fig:LogicGates} Logic gate equivalent of $\boldsymbol{(x \land y) ~\lor \sim y}$.}
	\end{center}
\end{figure}

%------------------------------------------------------------------------------
% 2 f--------------------------------------------------------------------------
%------------------------------------------------------------------------------

\subquestion

\begin{subsubquestions}
	
%-----------------------------------------------------------------------------

\subsubquestion
The Boolean expression for the given logic gate circuit is:
\begin{equation}
	\boldsymbol{\sim (a \lor b)}\,.
\end{equation}

%-----------------------------------------------------------------------------

\subsubquestion
The Boolean expression for the given logic gate circuit is:
\begin{equation}
	\boldsymbol{\sim a ~\land \sim b}\,.
\end{equation}

%-----------------------------------------------------------------------------

\subsubquestion

We can construct the following truth table to represent the circuits above.
\begin{table}[h]
	\centering
	\begin{tabular}{|c|c|c|c|c|c|c|}
		\hline
		$\boldsymbol{a}$ & $\boldsymbol{b}$ & $\boldsymbol{\sim a}$ & $\boldsymbol{\sim b}$ & $\boldsymbol{a \lor b}$ & $\boldsymbol{\sim (a \lor b)}$ & $\boldsymbol{\sim a \land \sim b}$ \\
		\hline
		0 & 0 & 1 & 1 & 0 & 1 & 1 \\
		0 & 1 & 1 & 0 & 1 & 0 & 0 \\
		1 & 0 & 0 & 1 & 1 & 0 & 0 \\
		1 & 1 & 0 & 0 & 1 & 0 & 0 \\
		\hline
	\end{tabular}
	\caption{\label{2009:q2:tab:TruthTab2} Truth table of $\boldsymbol{\sim (a \lor b)}$ and $\boldsymbol{(\sim a ~\land \sim b)}$.}
\end{table}

From \rtab{2009:q2:tab:TruthTab2}, we see that $\boldsymbol{\sim (a \lor b)}$ and $\boldsymbol{(\sim a ~\land \sim b)}$ are logically equivalent.

\end{subsubquestions}

\end{subquestions}


